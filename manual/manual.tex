\documentclass[a4paper,10pt]{scrbook}
\usepackage{amssymb,amsmath,array,cite,graphicx,a4wide,ae,subfig}

\pagestyle{plain}
\bibliographystyle{jcp}
\setlength{\parindent}{0in}
\addtolength{\parskip}{\baselineskip}

\newcommand{\ogo}{\textsc{ogolem}}

\author{Johannes M. Dieterich and Bernd Hartke}
\title{The \textsc{ogolem}.org Manual}

\begin{document}
\maketitle

\tableofcontents

%TODO  2625.4995 to 2625.49962


\chapter{General remarks}
\section{This manual}
Is alpha and might not include everything you would like to know. If this is
true, contact \ogo{} via webpage and mail. If you happen to find any kind of
error or uncertainty, please do also contact us.

\section{Licensing}
\ogo{} is distributed under the following license
\begin{verbatim}
Copyright (c) 2009-2020, J. M. Dieterich and B. Hartke
All rights reserved.

Redistribution and use in source and binary forms, with or without
modification, are permitted provided that the following conditions are met:

    * Redistributions of source code must retain the above copyright
      notice, this list of conditions and the following disclaimer.

    * Redistributions in binary form must reproduce the above copyright
      notice, this list of conditions and the following disclaimer in the
      documentation and/or other materials provided with the distribution.

    * All advertising materials mentioning features or use of this software
      must display the following acknowledgement:

      This product includes software of the ogolem.org project developed by
      J. M. Dieterich and B. Hartke (Christian-Albrechts-University Kiel,
      Germany) and contributors.

    * Neither the name of the ogolem.org project, the University of Kiel
      nor the names of its contributors may be used to endorse or promote
      products derived from this software without specific prior written
      permission.

THIS SOFTWARE IS PROVIDED BY THE AUTHOR(S) ''AS IS'' AND ANY
EXPRESS OR IMPLIED WARRANTIES, INCLUDING, BUT NOT LIMITED TO, THE IMPLIED
WARRANTIES OF MERCHANTABILITY AND FITNESS FOR A PARTICULAR PURPOSE ARE
DISCLAIMED. IN NO EVENT SHALL THE AUTHOR(S) BE LIABLE FOR ANY
DIRECT, INDIRECT, INCIDENTAL, SPECIAL, EXEMPLARY, OR CONSEQUENTIAL DAMAGES
(INCLUDING, BUT NOT LIMITED TO, PROCUREMENT OF SUBSTITUTE GOODS OR SERVICES;
LOSS OF USE, DATA, OR PROFITS; OR BUSINESS INTERRUPTION) HOWEVER CAUSED AND
ON ANY THEORY OF LIABILITY, WHETHER IN CONTRACT, STRICT LIABILITY, OR TORT
(INCLUDING NEGLIGENCE OR OTHERWISE) ARISING IN ANY WAY OUT OF THE USE OF THIS
SOFTWARE, EVEN IF ADVISED OF THE POSSIBILITY OF SUCH DAMAGE.
\end{verbatim}
in the hope to help people crack problems and enable interesting science. We do 
ask you to cite our papers
introducing the framework and techniques you use in your
publications\cite{ogolem_first,lj_reparam,ogo_switches, 
kana_dimer,benchmark,water_pccp,gdm,water_small,cage1,cage2,earth}. Depending 
on what is used in or through \ogo{} additional citations to papers describing 
these methods may be required.

We encourage all collaborations, external implementations, usages of \ogo{} as 
a library for any purposes. We are open for discussions regarding uses of 
\ogo{} under \texttt{developers@ogolem.org}.

\section{External libraries and code contributions}
\ogo{} is grateful for the following codes it uses or took inspiration from (in
random oder):
\begin{itemize}
 \item Atomdroid by J. Feldt, R. A. Mata and J. M. Dieterich (proprietary
license)
 \item Lionbench by M. Kammler and J. M. Dieterich (4-clause BSD)
 \item L-BFGS from RISO by J. Nocedal and R. Dodier (Apache/public domain)
 \item BOBYQA from the Apache Software Foundation (Apache)
 \item NEWUOA code by J. Feldt and J. M. Dieterich, original by M. J. D. Powell
(public domain)
 \item JAMA
 \item Apache Commons Math
 \item Scala
 \item openJDK
 \item Numal
 \item Arpack/Netlib
 \item SL4J
 \item JGraphT
 \item mpiJava
 \item phenix v0-42
 \item Tinker
\end{itemize}    

\section{Financial Acknowledgements}
The authors would like to acknowledge the following affiliations:
\begin{itemize}
 \item J. M. Dieterich
 \begin{itemize}
   \item \textit{2008-2010:}\\Institute for Physical 
Chemistry\\Christian-Albrechts-University Kiel\\Germany, Europe
   \item \textit{2010-2012:}\\Institute for Physical 
Chemistry\\Georg-August-University G\"ottingen\\Germany, Europe
   \item \textit{2015-2018}\\Department of Mechanical\&Aerospace 
Engineering\\Princeton University\\USA
 \end{itemize}
 \item B. Hartke
 \begin{itemize}
   \item \textit{2008-now:}\\Institute for Physical 
Chemistry\\Christian-Albrechts-University Kiel\\Germany, Europe
 \end{itemize}
\end{itemize}


\section{The \ogo{}-jar}
\ogo{} is distributed as a single jar file. By invoking
\begin{verbatim}
java -jar ogolem.jar
\end{verbatim}
followed by a single mandatory flag deciding which program part to use. These
flags are
\begin{verbatim}
   --adaptive
     for global parameter optimization
   --beswitched
     for extracting intermediate switch optimization results
   --clusteroverlap
     for checking if two cluster structures are identical
   --core
     for MPI parallelized cluster structure optimization
   --corrfunc
     to compute correlation functions from MD trajectories
   --help
     to display this overview
   --md
     for running simple molecular dynamics
   --molecules
     to optimize molecules with respect to a or multiple properties
   --parameters
     for extracting intermediate parametrization results
   --rmiserver
     for starting an OGOLEM RMI server
   --rmiproxy
     for starting an OGOLEM RMI proxy
   --rmiclient
     for starting an OGOLEM RMI client
   --rmithreader
     for starting an OGOLEM RMI threading client
   --shmem
     for thread-based cluster structure optimization
   --switches
     for thread-based molecular switch optimization
   --tests
     for testing and development purposes
   --version
     for the OGOLEM snapshot version
\end{verbatim}
Please note here that you should have installed a Java runtime environment (JRE)
complying at least to the JAVA7 standard. Additionally, for above command to
work, the java binary should be in your path, otherwise you will need to
explicitly specify the path to it.

If you have a choice, use the newest openJDK (currently openJDK9) in server mode
(the latter is really important for general performance and seems to be the
default for most JREs for most UNIX incarnations). Also, the author(s) want to
note that they could not see any performance gain on standard hardware
with proprietary JREs like JRockit or IBM JDK.

\section{General notes}
\subsection{Units}
All units within \ogo{} are (if applicable) atomic units. As a secondary system,
we provide SI units as a convinience. Therefore, input and output is (unless 
explicitly
stated otherwise) in atomic units. Obviously, this only applies for the case of 
data
having a physical meaning, otherwise input/output is system dependent and 
typically
unit-free.

\section{Bugs and problems}
Since \ogo{} is a relatively new program that is still changing rapidly, we
unfortunately can only guarantee that you will probably hit bugs and problems.

In case you hit a bug:
\begin{itemize}
  \item Relax.
  \item Mail \texttt{developers@ogolem.org} describing the problem and how to 
reproduce it. Please attach input and error messages (in particular: stack 
traces).
  \item Wait for the bugfix which will be rolled automatically in the next 
autobuild.
\end{itemize}



\chapter{Global optimization of clusters}
For the global optimization of cluster structures on a shared memory basis,
invoke
\begin{verbatim}
java -jar ogolem.jar --shmem input.ogo 2
\end{verbatim}
where \texttt{input.ogo} is your input file, needing to have \texttt{.ogo} as a
suffix, followed by the number of threads you want to use, 2 in this case.

If you want to make use of the MPI frontend, please contact the authors for help
since this is a non-trivial procedure due to the shortcomings of the MPI
standard. In general, you are better off with shared memory parallelization.

See Chapter.~\ref{rmi} for a JVM-based substitute for MPI with some more
advanced distribution of computing tasks and stability. However, this feature
requires some setup! Please contact the authors for help with this.

\section{General input structure}
Every input file needs to start with the line
\begin{verbatim}
###OGOLEM###
\end{verbatim}
for \ogo{} to recognize that this is supposed to be a valid input file. Now you
can add some input keywords and the building block input. This includes the
actual cluster building blocks (atoms or molecules) as well as (if applicable)
spins and charges.

In case of explicit degree of freedom searches or constraints, more input blocks
are necessary.

\section{Building block input}
In general, any combination of any building blocks is possible in principle. The
only thing you should have for the easiest case is the number of building blocks
and xyz-files of all your units.

The input starts with an opening tag followed by the number of building blocks
in the next line, 5 in this example
\begin{verbatim}
<GEOMETRY>
NumberOfParticles=5
\end{verbatim}
The building blocks (named molecules) are now put there in a similar form after
each other, for example
\begin{verbatim}
<MOLECULE>
MoleculePath=water.xyz
</MOLECULE>
<MOLECULE>
MoleculePath=ethanol.xyz
</MOLECULE>
<MOLECULE>
MoleculePath=citronacid.xyz
</MOLECULE>
<MOLECULE>
MoleculePath=quinine.xyz
</MOLECULE>
<MOLECULE>
MoleculePath=water.xyz
</MOLECULE>
\end{verbatim}
The number of molecules and the \texttt{NumberOfParticles} needs to agree,
otherwise an exception will be raised.

If the molecule consists of a single atom, another input option might be easier
\begin{verbatim}
<MOLECULE>
Na;0.0;0.0;0.0
</MOLECULE>
\end{verbatim}
where the 0.0s are internal xyz values.

In case of the molecule having internal degrees of freedom that should be
explicitly optimized, the molecule path must point to a z-Matrix and the
first line in the definition is the flexy keyword
\begin{verbatim}
<MOLECULE>
flexy
MoleculePath=methanol.zmat
</MOLECULE>
\end{verbatim}
It should be noted that the zmat-format chosen is of form
\begin{verbatim}
6

C
O  1.4   1
H  0.95  2  109.471 1
H  1.089 1  109.471 2  120.0 3
H  1.089 1  109.471 2  240.0 3
H  1.089 1  109.471 2  0.0   3
\end{verbatim}

A bond matrix for the building block may be specified by the user through the
\texttt{BondMatrix=} keyword. It must only be used following the
\texttt{MoleculePath=} keyword in both flexible and non-flexible input. The
path must point to a file containing the bond matrix definition, either as a
complete matrix or as a bond list. This input option must be used if the molecule
does not only contain single bonds or if the automatic bond detection (see output)
fails to correctly assign bonds. Examples:

\begin{verbatim}
<MOLECULE>
flexy
MoleculePath=water.zmat
BondMatrix=waterbonds.mat
</MOLECULE>
\end{verbatim}

where \texttt{waterbonds.mat} (name must not end with \texttt{.list} and is
otherwise not important) contains

\begin{verbatim}
1 1 1
1 1 0
1 0 1
\end{verbatim}

where 1 denotes a single bond, 0 denotes no bond. The bonds on the diagonal are 
arbitrary. Allowed bonds are: 0 for no bond, 1 for a single, 2 for a double, and
3 for a triple bond. We will add more bond types as backends make use of them.

Another example usage:
\begin{verbatim}
<MOLECULE>
MoleculePath=water.zmat
BondMatrix=waterbonds.list
</MOLECULE>
\end{verbatim}

where \texttt{waterbonds.list} (files containing bond lists must end in
\texttt{.list}) contains

\begin{verbatim}
0 1 1
0 2 1
\end{verbatim}

do denote a single bond between atoms 0 and 1 and 0 and 2 in the molecule.

Obviously the list-based input is less verbose/redundant and should be used in
general. Please note once more that the counting starts with zero! Both
list-based and matrix-based input may be used in both the flexible and
non-flexible context.

The geometry definition is closed using a line with a 
\begin{verbatim}
</GEOMETRY>
\end{verbatim}
tag. Please note that the numbering of molecules for all subsequent inputs
starts from 0.

This is followed by, if applicable, input concerning spins and charges. Charges
are defined as follows
\begin{verbatim}
<CHARGES>
MOLECULENO;ATOMNO;CHARGE
</CHARGES>
\end{verbatim}
for example
\begin{verbatim}
<CHARGES>
0;0;1
1;3;-2
</CHARGES>
\end{verbatim}
Above example would put a single positive charge at the first atom of the first
molecule and two negative ones at the fourth atom of the second molecule. It is
important to note here that, since arrays in the Java language start with 0 and
not 1 as in some prehistoric languages, our counting also starts at 0.

A similar input structure exists for the spins in the system, namely
\begin{verbatim}
<SPINS>
MOLECULENO;ATOMNO;SPIN
</SPINS>
\end{verbatim}
for example
\begin{verbatim}
<SPINS>
0;2;1
1;3;-1
</SPINS>
\end{verbatim}
where the above example would put a single spin up at atom 0 of molecule 2 and a
spin down at atom 3 of molecule 1.

Should there be no spins or charges in the studied system, no spins or charges
block is necessary.

Note for larger systems build from only a few building blocks: one can add
inside the \texttt{<MOLECULE>} block the \texttt{MoleculeRepetitions=} keyword
and set it to the number of species of this type. Please note two things: 1) the
\texttt{NumberOfParticles=} setting must still match the total number of species
and 2) the spins/charges/... are automatically reproduced for all species of one
type. An example:

\begin{verbatim}
GEOMETRY>
NumberOfParticles=24
<MOLECULE>
MoleculeRepetitions=23
MoleculePath=water.xyz
</MOLECULE>
<MOLECULE>
MoleculeRepetitions=1
MoleculePath=water.xyz
</MOLECULE>
</GEOMETRY>
<CHARGES>
0;0;0
1;0;-1
</CHARGES>
\end{verbatim}
would put 23 neutral water molecules and 1 negatively charged one into the
cluster. The first charge line is not needed.

\section{Constraints and explicit degrees of freedom}
Constraints can be specified based on Cartesian coordinates. Although \ogo{}
internally allows for arbitrary constraints to be specified, the backend
carrying out the local optimization might not like that. In this case, a
warning should appear.

In general, the constraints are specified similar to the spins and charges,
through tags of form
\begin{verbatim}
<CONSTRAINTS>
MOLECULENO;ATOMNO;COORDINATE
</CONSTRAINTS>
\end{verbatim}
where the cartesian coordinate is chosen through an integer value (0 is x, 1 is
y and 2 is z). Additionally, special input exists to constrain a whole
molecule or all x, y or z coordinates of a molecule. Some example are
\begin{verbatim}
<CONSTRAINTS>
0;all
1;allx
2;ally
3;allz
4;1;0
4;2;1
5;0;2
</CONSTRAINTS>
\end{verbatim}
The first one constraining all cartesian coordinates of molecule 0, the second
only the x coordinates of molecule 1, the third the y coordinates of molecule 2
and the fourth the z coordinates of molecule 3. Then the x coordinate of atom 1
in molecule 4 is constrained, followed by y coordinate of atom 2 in molecule 4
and the z coordinate of atom 0 in molecule 5. Any logical combination of the
above described input possibilities is allowed and supported by \ogo{}.

Constraints are currently only supported in the following interfaces:
\begin{itemize}
  \item DFTB+\\
	in this case the constraint is actually a restraint and the convergence
threshold needs to be increased since the gradient along the restraint is
included in the calculation
  \item lotriff
  \item MOPAC
  \item Orca
\end{itemize}
Again, if your preferred level of theory does not support constraints in \ogo{}
and you would like to use that feature, contact us (preferably with reference
input for that particular program) or send us patches.

Explicit degrees of freedom are a rather untested feature, therefore one
unfortunaltely needs to expect some rough edges. Also it should be noted that
you cannot put constraints and explicit degrees of freedom on the same molecule.

As said before, you need to enable flexibility on the molecule with explicit
degrees of freedom in the building block input and need to speify a z-Matrix
instead of cartesian coordinates. When this has been ensured, the specification 
of explicit
degrees of freedom is relatively straightforward. Again, it includes
definitions of type
\begin{verbatim}
<DOF>
MOLECULENO;ATOMNO;COORDINATE
</DOF>
\end{verbatim}
where the coordinate is a choice for the coordinate in the z-Matrix
(i.e., distance, angle, dihedral). For example
\begin{verbatim}
<DOF>
0;1;dihedral
2;3;distance
4;2;angle
</DOF>
\end{verbatim}
where the first line would make the dihedral of atom 1 in
molecule 1, the second would make the bond distance 
of atom 3 in molecule 2 and the third would make the angle of
atom 2 in molecule 4 an explicit DoF. Alternatively, one may use the syntax
\begin{verbatim}
<DOF>
0;all
\end{verbatim}
to enable explicit degrees of freedom for all DoFs of molecule 0.

Please let us know about any experiences with explicit DoFs, may they be good
or bad.

\section{Input keywords}
There are a lot of possible tunables in the \ogo{} framework. Some of them are
more important than others and should actually be adjusted to the studied
system.

\begin{itemize}
  \item \texttt{DebugLevel=}
	the debug level: controls additional output of some code parts. Please
set to 1 or higher \textsc{first} if you ever have unexpected failures or problems.
Defaults to 0.
  \item \texttt{BlowBondDetect=}\\
	a collision is detected if any of the non-bonded atoms in the structure
are getting closer to each other than this blow factor times their added radii.
Defaults to 1.2, which is a rather conservative choice.
  \item \texttt{BlowFacDissoc=}\\
	the blow factor for the dissociation detection. Defaults to 3.0 which
should be decreased for LJ systems (e.g. 2.0 seems to be a reasonable
choice). Notice that this setting has changed since the last version!
  \item \texttt{BlowInitialBonds=}\\
	the blow factor for the initial bond detection. Since these bonds are
the basis for both the post sanity check and the CD, the factor normally should
be equal to \texttt{BlowBondDetect=}. Also defaults to 1.2.
  \item \texttt{CellSize=}\\
	how big the initial cell should be for the initial structure
generation. Defaults to a \texttt{9;9;9} bohrs which is too small for bigger
systems.
  \item \texttt{InitialFillAlgo=}\\
	how the intial individuals should be generated. Defaults to 
\texttt{packrandomly}, which is
a randomized packing init w/o considering molecular size. If that should be
yielding too densly packed (or spherical) structures,
change to \texttt{randomwithoutdd}, which is a randomized init w/o dissociation 
detection or \texttt{randomwithdd}, which
is a randomized init w/ DD.
  \item \texttt{MaxIterLocOpt=}\\
	how many iterations the local optimization is allowed to carry out.
Defaults to 2000.
  \item \texttt{NumberOfGlobIterations=}\\
	how many global optimization steps should be carried out. Defaults to
1000 which is just sufficient for the most trivial cases. This value must be
increased for production jobs!
  \item \texttt{PoolSize=}\\
	one of the most important keywords. Specifies how big the genetic pool
is allowed to be. Defaults to 100 individuals.
  \item \texttt{AcceptableFitness=}\\
        \label{page1:acceptableFitness}
	what fitness is acceptable and once reached makes the threading version
	of \ogo{}
finish. Set to negative infinity per default and should only be set for
benchmarking purposes and only if you are know your way around the \ogo{}
program code extremely well.
  \item \texttt{CollisionDetection=}\\
	which collision detection to use. Defaults to \texttt{simplepairwise}, 
which is a pairwise
collision detection engine. Keep.
  \item \texttt{CrossoverPossibility=}\\
	chances for crossover in the global optimization. Defaults to 1.0
(100\%).
  \item \texttt{DissociationDetection=}\\
	which dissociation detection to use. Defaults to \texttt{dfs}, which is 
a
recursive quadratic scaling algorithm. Keep.
  \item \texttt{DoGeometryNiching=}\\
	if a niching algorithm should be used for this run. Defaults to false.
  \item \texttt{GeneticRecordBufferSize=}\\
        number of steps till a summary of the genetic records are flushed to
the output file. Defaults to 1000.
  \item \texttt{GeneticRecordsToSerial=}\\
	number of steps till the genetic records are written to disk. Defaults
to 1000.
  \item \texttt{GeometriesToSerial=}\\
	number of steps till the pool is written to disk. Defaults to
99, which is too small for big systems with a lot of steps and causes an 1/O
bottleneck and should be increased then.
  \item \texttt{SerializeAfterNewBest=}\\
	if the pool shall be serialized whenever a new best individual has been
added. Defaults to true.
  \item \texttt{GeometryChoice=}\\
	how the parent individuals should be selected. Three options exist:
	\begin{itemize}
	 \item \texttt{random:} \\ both parent individuals are randomly
           selected. With the additional option \texttt{stepat=X.X} (with
           $0\leq $\texttt{X.X}$ \leq 1$) this is turned into a step-function
           selector for the first X.X percent of the sorted pool. With
           \texttt{stepat1=X.X,stepat2=Y.Y}, you can assign
           different step functions to parent1 and parent2. When no steps
           are given/desired, this corresponds to \texttt{X.X}=\texttt{Y.Y}=1
	 \item \texttt{fitnessrankbased:} \\  one or both parents are selected
	 based on a Gaussian over their ranked fitness. Valid, comma separated
	 options:
	 \texttt{bothfitness}: choose both parents fitness based (default: no),
	 \texttt{gausswidth=X.X} sets the width of the Gaussian to X.X (default:
	 0.5). 
         If you want to select parent1 and parent2 with Gaussians of different
         width, simply use \texttt{gausswidth1=X.X} and
         \texttt{gausswidth2=Y.Y}. 
         Our default selector with a Gaussian width of 0.05.\\
	 \textbf{THIS SETTING WILL VERY LIKELY CHANGE IN THE VERY NEAR FUTURE!}
	 \item \texttt{fitnessvaluebased:} \\  one or both parents are selected
	 based on their fitness value. Valid, comma separated options:
	 \texttt{bothfitness}: choose both parents fitness based (default: no),
	 \texttt{usegausshape}: use a Gaussian shaped random number for the
	 selection (default: linear), \texttt{tailprob=X.X} sets the tail 
probability
	 to X.X (default: 0.1). The tail probability is defined as the 
probability
	 for the last element in the pool to be chosen with the head probability
	 (i.e., the probability for the fittest individual) being 1.0.
	\end{itemize}

  \item \texttt{GlobOptTries=}\\
	how many tries the global optimization algorithm should carry out at
any step that cause either collision and/or dissociation before proceding with
the next parents. Defaults to 200 which is suitable for bigger systems with
more degrees of freedom.
  \item \texttt{GrowCell=}\\
	whether the cell of the starting structure is allowed to grow. Defaults
to \texttt{false}. If you are having problems with the randomized structure
preparation, specify a bigger cell and/or set this to \texttt{true}.
  \item \texttt{IntLocOptMaxStep=}\\
	the maximum step length for the internal BFGS. Defaults to 1E0 which
seems to be reasonable.
  \item \texttt{IndividualsPerNicheAtMax=}\\
	how many individuals are maximally allowed per niche, if niching is
enabled. Defaults to 50.
  \item \texttt{MaxBondStretch=}\\
	how much a bond is allowed to be streched as a factor. Defaults to 1.3
which seems to be reasonable.
  \item \texttt{MolecularCDInInit=}\\
	whether there should be a molecular collision detection in the geometry
initialization. Defaults to true.
  \item \texttt{MolecularCoordinateMutation=}\\
	how the coordinates should be mutated. Defaults to \texttt{some}, which 
always
mutates a couple of them at the same time. \texttt{one} just mutates a single 
one at any
time.
  \item \texttt{MutationPossibility=}\\
	chances for mutation of an individual during the global optimization.
Defaults to 0.05 (5\%).
  \item \texttt{NichingAddsToStats=}\\
	how many individuals shall be added to niches (if enabled) before
statistics are printed to command line. Defaults to 50.
  \item \texttt{PostSanityCD=}\\
	if post-local optimization sanity checks are enabled, should the
resulting structure be checked for collisions? Defaults to false.
  \item \texttt{PostSanityDD=}\\
	if post-local optimization sanity checks are enabled, should the
resulting structure be checked for dissociation? Defaults to false.
  \item \texttt{PostSanityCheck=}\\
	whether the structure should be checked for sanity after the local
optimization. Defaults to \texttt{true} which is reasonable for most cases.
  \item \texttt{PrintGeomsBeforeLocOpt=}\\
	whether the geometries should be printed to the command line after the
global optimization operators, before the local optimization. Useful for setup
purposes. Defaults to false.
  \item \texttt{RatioExplDoFInit=}\\
	what fraction of the explicit degrees of freedom should be randomized in
the initialization. Applied per molecule with explicit DoFs. Defaults to 1.0,
which might be too big for big molecules with a lot of explicit DoFs. If \ogo{}
seems to be stalled in the init, this might be a reason.
  \item \texttt{SeedingPath=}\\
	if the initial pool should be seeded with some structures set this to
the folder containing these structures in the xyz-file format. \ogo{} will try
to read in all files in this folder so there must only be xyz files present.
  \item \texttt{ThreshLocOptCoord=}\\
	which threshold is acceptable for the coordinates in the local
optimization. Defaults to 1E-8 which should be kept, if one is uncertain.
  \item \texttt{ThreshLocOptGradient=}\\
	which threshold is acceptable for the gradient in the local
optimization. Defaults to 1E-8 which should be kept, if one is uncertain.
\end{itemize}

Some special keywords for restarts are
\begin{itemize}
  \item \texttt{restart=}\\
      Whether this run is supposed to be restarted or not. Defaults to
\texttt{false}. If set to \texttt{true}, some other keywords need to be set as
well
  \item \texttt{InitialFill=}\\
      Defaults to \texttt{true} and defines whether we should make an initial
fill. Will be automagically set by the restart keyword.
  \item \texttt{InitialLocOpt=}\\
      Defaults to \texttt{true} and defines whether the read in pool should be
locally optimized in a first step.
  \item \texttt{GlobalOptimization=}\\
      Defaults to \texttt{true} and defines whether there should be any global
optimization steps.
  \item \texttt{NamingOffset=}\\
      which ID to start from. Increase to the number of steps of your previous
run.
\end{itemize}
However, we have found it superior to use (part of) the results of a previous
run as seeds for a subsequent run. Hence, these options may vanish in the
future.

Within the block of keywords, one can comment lines using either \texttt{\#} or
\texttt{//}.

\section{Local optimization and external backends}
\ogo{} supports a variety of program packages through interfaces. We hope that
this satisfies most users. If there should be a (preferably free) program that
you want to see supported and/or the methods listed here to do not satisfy you,
feel free to write a mail to developers@ogolem.org .

In general, most methods are easily called through input of form
\begin{verbatim}
LocOptAlgo=PACKAGE:METHOD
\end{verbatim}


In general, there are thoughts to change this input for something more flexible
so future versions might behave different. Also it should be noted that of
course \ogo{} might not support any revision of the programs mentioned if their
input changes. If you should hit such a case and are not working with some
ancient version (e.g. there is for sure no PM6 in MOPAC7), please contact us
and wait for a fix.

%TODO document InitialLocOptAlgo
%TODO document tinker:ubercustom

\subsection{Interface to ADFSuite}
An experimental interface to the ADFSuite of programs exists. It is chosen by
setting \texttt{LocOptAlgo=adf:JOBKEY}, where \texttt{JOBKEY} can either be a 
build-in job identifier, such as \texttt{GO} or \texttt{DFTB-GO}, or the file
name of an adf stub file, e.g., \texttt{mylocopt.adf}. In the later case, this 
file name
must end in \texttt{.adf} and be present in the working directory.

A stub file can be created using, e.g., \texttt{adfjobs} and tuned to the 
particular
computational needs of the system under study.

\textbf{IMPORTANT: } \ogo{} will \textbf{NOT} adjust anything except for the 
geometry coordinates
in the job stub file. Therefore, a charged system or a system with spin does 
require creation
of a fitting job stub file. Likewise, convergence thresholds will not be 
propagated
from within \ogo{} to ADF and must be set in the job stub file!

Please note, you must have all the usual ADF/SCM environment variables properly 
setup,
e.g., \texttt{SCMLICENSE} and \texttt{ADFBIN}. This can be done by sourcing the
correct setup file for your platform. Obviously, a valid license for the codes 
to be run
is required.

\subsection{Interface to AMBER}
The interface to AMBER is called through \texttt{LocOptAlgo=amber}, followed by
the optional choice \texttt{:implicitwater}, if implicit solvation is wanted or
needed.

The \texttt{sander} executable needs to be in the path and named such.

Additionally, there needs to be a fitting amber parameter file named
\texttt{input.prmtop}, where \texttt{input} is the prefix of your
\texttt{input.ogo} file.

The AMBER interface in its current state is not suitable for molecule only
optimizations. This does currently not have any effect on \ogo{}.

\subsection{Interface to CP2K}
The interface to CP2K is called through \texttt{LocOptAlgo=cp2k:}, followed
by a sequence of keywords to specify the level of theory. Supported levels of
theory at the moment are
\begin{itemize}
  \item \texttt{B3LYP}
  \item \texttt{BLYP}
  \item \texttt{BP}
  \item \texttt{HCTH120}
  \item \texttt{OLYP}
  \item \texttt{PADE}
  \item \texttt{PBE}
  \item \texttt{PBE0}
  \item \texttt{TPSS}
\end{itemize}

The \texttt{cp2k.sopt} executable needs to be in the path and named such.

Additionally, an \texttt{input-cp2k.aux} file is required, where \texttt{input}
is the prefix of your \texttt{ogo}-file. This auxiliary file needs to start
with a line
\begin{verbatim}
###OGOLEMAUX###
\end{verbatim}
followed by a range of lines specifying in three whitespace-separated columns
which atom should use which basis set and which potential. These potentials and
basis sets need to be present in the directory.

\subsection{Interface to CPMD}
The interface to CPMD is called through \texttt{LocOptAlgo=cpmd:}, followed
by a sequence of keywords to specify the level of theory. Supported levels of
theory at the moment are
\begin{itemize}
  \item \texttt{SONLY}
  \item \texttt{LDA}
  \item \texttt{BONLY}
  \item \texttt{BP}
  \item \texttt{BLYP}
  \item \texttt{XLYP}
  \item \texttt{GGA}
  \item \texttt{PBE}
  \item \texttt{PBES}
  \item \texttt{REVPBE}
  \item \texttt{HCTH}
  \item \texttt{OPTX}
  \item \texttt{OLYP}
  \item \texttt{TPSS}
  \item \texttt{PBE0}
  \item \texttt{B1LYP}
  \item \texttt{B3LYP}
  \item \texttt{X3LYP}
  \item \texttt{CUSTOM}
\end{itemize}

The \texttt{cpmd.x} executable needs to be in the path and named such.

Additionally, an \texttt{input-cpmd.aux} file is required, where \texttt{input}
is the prefix of your \texttt{ogo}-file. This auxiliary file needs to start
with
\begin{verbatim}
###OGOLEMAUX###
70
15;15;15
\end{verbatim}
Where the second line is the cutoff in Rydberg and the third line is the cell
size to be used (if line is empty it will be estimated for each geometry, based
on the assumption of a clustered system!). This is followed by a range of lines
specifying in three whitespace-separated columns
which atom should use which pseudopotential and which lmax. The pseudopotentials
need to be present in the directory. Please note that no extra lines are
allowed.

%TODO describe custom input


\subsection{Interface to DFTB+}
The interface to DFTB+ is called through \texttt{LocOptAlgo=dftb+:}, followed
by a sequence of keywords to specify some needed parameters for the local
optimization employed by DFTB+
\begin{itemize}
  \item \texttt{steepest}\\
	use a steepest descent algorithm
  \item \texttt{conjugate}\\
	use a conjugate gradient algorithm
  \item \texttt{steepest,scc}\\
	use a steepest descent algorithm with SCC approximation
  \item \texttt{conjugate,scc}\\
	use a conjugate gradient algorithm with SCC approximation
\end{itemize}

The DFTB+ executable needs to be in your path and named \texttt{dftb+}.

Additionally, you need to make sure to have all Slater-Koster files for all
interactions in your system in the directory your \texttt{ogo}-input file is.
This is needed for \ogo{} using the SK-files for the DFTB+ input.

\subsection{Interface to GROMACS}
The interface to GROMACS is called through \texttt{LocOptAlgo=gromacs}. It is
followed by an optional choice of the QM backend to be used. Choices are
\begin{itemize}
  \item empty/whitespace\\
	no QM backend
  \item \texttt{:mopac}\\
	use MOPAC7 as a backend. Please note that MOPAC7 only provides AM1 and
PM3.
  \item \texttt{:orca}\\
	use Orca as a backend.
  \item \texttt{:gaussian}\\
	use Gaussian as a backend.
  \item \texttt{:gamess-uk}\\
	use Gamess-UK as a backend.
\end{itemize}
In all cases, the GROMACS binaries need to be recompiled according to the
GROMACS documentation depending on the chosen backend.

The path to the \texttt{grompp} and \texttt{mdrun} excecutables needs to be
exported through the environment variables \texttt{OGO\_GROMACSMPP} and
\texttt{OGO\_GROMACSCMD}, e.g. in bash syntax
\begin{verbatim}
export OGO_GROMACSMPP=$HOME/software/gromacs/grompp
export OGO_GROMACSCMD=$HOME/software/gromacs/mdrun
\end{verbatim}

Additionally, the files 
\begin{itemize}
\item \texttt{gromacs-topology.top},
\item \texttt{gromacs-parameters.mdp} and
\item \texttt{gromacs-index.ndx}
\end{itemize}
need to present and correct for the system under study. These files are also
responsible for all GROMACS related configuration, e.g. constraints and QM/MM
partitioning. The file
\begin{verbatim}
INPUTNAME-gromacs.aux
\end{verbatim}
needs to contain in the first line the identifier \texttt{\#\#\#OGOLEMAUX\#\#\#}
followed by the first half of a \texttt{.g96} file of the system (residue and
atom-IDs).

If one uses Orca as a QM-backend for Gromacs, one also needs to export the path
to the orca excecutables and the basename of the optimization job. This is e.g.
in bash syntax accomplished through
\begin{verbatim}
export ORCA_PATH=$HOME/software/orca
export BASENAME=topol
\end{verbatim}
where the \texttt{BASENAME} is always \texttt{topol}! Additionally, the file
\texttt{gromacs-orca.aux} needs to contain the input keywords, as described in
the GROMACS manual.

\subsection{Interface to lotriff}
Lotriff is an in-house development by Bernd Hartke. If you are in desperate need
of this FF engine, contact him via mail. If you obtained the \texttt{lotriff}
executable you can use \texttt{LocOptAlgo=lotriff}.

The path to the lotriff executable needs to be exported in the
\texttt{OGO\_LOTRIFFCMD} environment variable, e.g. in bash syntax
\begin{verbatim}
export OGO_LOTRIFFCMD=$HOME/software/lotriff/lotriff
\end{verbatim}
If this variable should be not set, \ogo{} will assume the binary to be in
your path.

Additionally, a correct \texttt{lotriff.in} file and a correct
\texttt{hetero.def} file need to be present in your directory, the constrained
building blocks are supposed to be specified first in the building block input.

A mapping of atom types in an auxiliary file \texttt{input-lotriff.aux}, where
input is the prefix of your input file name, needs to be present. First line of
this file needs to be
\begin{verbatim}
###OGOLEMAUX###
\end{verbatim}
followed by a column of force field types as present in your lotriff input.

The lotriff interface in its current state is not suitable for molecule only
optimizations. This does currently not have any effect on \ogo{}.

\subsection{Interface to MNDO}
The interface to MNDO is called through \texttt{LocOptAlgo=mndo:}, followed by
a sequence of keywords to specify the method. Builtin support exists at the
moment for the following levels of theory
\begin{itemize}
  \item \texttt{mndod}
  \item \texttt{om1}
  \item \texttt{om2}
  \item \texttt{om3}
  \item \texttt{pm3}
  \item \texttt{am1}
  \item \texttt{mndoc}
  \item \texttt{mndo}
  \item \texttt{mindo3}
  \item \texttt{cndo2}
  \item \texttt{scc-dftb}
  \item \texttt{scc-dftbj}\\
	J\o{}rgensen DFTB
\end{itemize}
all of the above can make use of implicit solvation using COSMO when appending
\texttt{:cosmo}.

Since the MNDO program is rather unique when it comes to input and output, it
was ultimately decided to call not MNDO directly but instead to always call a
shell-script which just wraps that. This obviously just works on UNIX machines
but MNDO is also just available for them to our knowledge.

The shell script, named \texttt{mndo.sh}, needs to be in the path and include
something alike to the following
\begin{verbatim}
#!/bin/sh
$1 > mndo > $2
\end{verbatim}

\subsection{Interface to MOLPRO}
The interface to MOLPRO is called through \texttt{LocOptAlgo=molpro:}, followed
by a sequence of keywords to specify the method. Builtin support exists at the
moment for the following levels of theory
\begin{itemize}
  \item \texttt{hf/vdz}
  \item \texttt{b86/vdz}
  \item \texttt{mp2/avtz}
  \item \texttt{custom}\\
Requiring you to provide an input stub \texttt{YOURINPUT-molpro.aux} containing
all relevant input and instead of the geometry the key \texttt{OGOCOORDS}.
  \item \texttt{custom,nolocopt}\\
Just like above, an auxiliary file is required, with the \texttt{nolocopt} flag
signaling that no local optimization takes place, interesting for
chained local optimizations.
\end{itemize}

Additionally, the location of the MOLPRO executable needs to be exported in the
environment variable \texttt{OGO\_MOLPROCMD}, e.g. in bash syntax
\begin{verbatim}
export OGO_MOLPROCMD=$HOME/software/molpro2010.1/bin/molpro
\end{verbatim}
If this variable should be not set, \ogo{} will assume the binary to be in
your path.

Since MOLPRO is a closed-source program it is only supported on the platforms
the official MOLPRO release supports.

\subsection{Interface to MOPAC}
The interface to MOPAC is called through \texttt{LocOptAlgo=mopac:}, followed by
a sequence of keywords to specify the method. Builtin support exists at the
moment for the following levels of theory
\begin{itemize}
  \item \texttt{mndo}
  \item \texttt{am1}
  \item \texttt{pm3}
  \item \texttt{pm5}
  \item \texttt{pm6}
  \item \texttt{mndo,cosmo(ethanol)}
  \item \texttt{am1,cosmo(ethanol)}
  \item \texttt{pm3,cosmo(ethanol)}
  \item \texttt{pm5,cosmo(ethanol)}
  \item \texttt{pm6,cosmo(ethanol)}
  \item \texttt{mndo,cosmo(water)}
  \item \texttt{am1,cosmo(water)}
  \item \texttt{pm3,cosmo(water)}
  \item \texttt{pm5,cosmo(water)}
  \item \texttt{pm6,cosmo(water)}
\end{itemize}

Additionally, the location of the MOPAC executable needs to be exported in the
environment variable \texttt{OGO\_MOPACCMD}, e.g. in bash syntax
\begin{verbatim}
export OGO_MOPACCMD=$HOME/software/mopac2010/mopac
\end{verbatim}
If this variable should be not set, \ogo{} will assume the binary to be in
your path.

Since MOPAC is a closed-source program it is only supported on the platforms the
official MOPAC release supports.

\subsection{Interface to NAMD}

The interface to NAMD is called through \texttt{LocOptAlgo=namd}.

The \texttt{namd2} executable needs to be in the path and named such.

Additionally, there needs to be a fitting amber parameter file (yes, this is
not an error!) named \texttt{input.prmtop}, where \texttt{input} is the prefix
of your \texttt{input.ogo} file.

The NAMD interface in its current state is not suitable for molecule only
optimizations. This does currently not have any effect on \ogo{}.

\subsection{Interface to OpenBabel}
The interface to OpenBabel is called through \texttt{LocOptAlgo=openbabel:},
followed by a sequence of keywords to specify which force field to use.
Currently supported are
\begin{itemize}
  \item \texttt{ghemical}\\
	the usage of this FF is highly encouraged since it is finnish
technology ;-)
  \item \texttt{mmff94s}
  \item \texttt{mmff94}
  \item \texttt{uff}
\end{itemize}

The \texttt{obminimize} executable needs to be in the path and named so.

As a side note, the atom type recognition is done by OpenBabel automagically.
This might be error prone and it is recommended to be checked in advance to a
full global optimization.

\subsection{Interface to Orca}
The interface to Orca is called through \texttt{LocOptAlgo=orca:}, followed by a
sequence of keywords to specify the method. Builtin support exists at the
moment for the following levels of theory
\begin{itemize}
  \item \texttt{mndo}
  \item \texttt{am1}
  \item \texttt{pm3}
  \item \texttt{b3lyp/vdz}
  \item \texttt{bp86/svp}
  \item \texttt{bp86/tzvp}
  \item \texttt{ubp86/tzvp}\\
	forces unrestricted Kohn-Sham
  \item \texttt{b3lyp/6-31+g**}
  \item \texttt{mp2/aug-cc-pVDZ}
  \item \texttt{ccsd(t)/tzvpp}\\
	only use this if you have way too much computing time
  \item \texttt{custom}\\
Requiring you to provide an input stub \texttt{YOURINPUT-orca.aux} containing
all relevant input before the geometry block. Please note, that this input needs
to contain the constraints, if wanted.
  \item \texttt{custom,nolocopt}\\
Just like above, an auxiliary file is required, with the \texttt{nolopt} flag
signaling that no local optimization takes place, interesting for
chained local optimizations.
\end{itemize}

Additionally, the location of the orca executable needs to be exported in the
environment variable \texttt{OGO\_ORCACMD}, e.g. in bash syntax
\begin{verbatim}
export OGO_ORCACMD=$HOME/software/orca_v2.8.0/orca
\end{verbatim}
If this variable should be not set, \ogo{} will assume the binary to be in
your path.

Since Orca is a free, albeit closed-source program, it is only supported on the
platforms the official Orca release supports (at the moment: Linux, Solaris,
Windows).

\subsection{Interface to Tinker}
The interface to Tinker is called through \texttt{LocOptAlgo=tinker:}, followed
by a sequence of keywords to specify some needed parameters for the local
optimization employed by Tinker
\begin{itemize}
  \item \texttt{minimize}\\
	use Tinkers minimize program for local optimization
  \item \texttt{newton}\\
	use Tinkers newton program for local optimization
  \item \texttt{optimize}\\
	use Tinkers optimize program for local optimization
  \item \texttt{custom}\\
	use a custom command and options for the local optimization
\end{itemize}

For the earlier three, the Tinker executables need to be in your path and named
\texttt{minimize}, \texttt{newton} and \texttt{optimize}. For the last option,
one needs to export the system variables \texttt{OGO\_TINKERCMD} and
\texttt{OGO\_TINKEROPT}, e.g. in bash syntax
\begin{verbatim}
export OGO_TINKERCMD=$HOME/software/tinker_patched/optimize
export OGO_TINKEROPT="0.01"
\end{verbatim}

Additionally, you need to have a file called \texttt{input-tinker.aux}, where
\texttt{input} is the prefix of your input file in the same directory as the
\texttt{ogo}-file. This file needs to start with a line
\begin{verbatim}
###OGOLEMAUX###
\end{verbatim}
followed by a line with the name of the tinker parameter file to use (e.g.
\texttt{mm3.prm}), which also needs to be in the directory. Then the force
field types for all atoms in your systems (same order as in the building block
definition). The first column being some arbirtray identifier, followed by
whitespace and the force field type ID as in the parameter file.

Occasionally, the setup described above does not work but results in
error messages like this:
\begin{verbatim}
WARNING: Error in Tinker: optimize locopt. Returning non-optimized geometry. 
org.ogolem.core.ConvergenceException: Tinker has a problem (local optimization).
\end{verbatim}
If this happens, first make sure that Tinker in stand-alone mode (w/o \ogo{})
can really do the intended calculations, using the input files automatically
generated by \ogo{}. If Tinker does work in stand-alone mode, the above error
may be related to system I/O issues on certain Linux systems. As a
possible fix, you are then encouraged to try this workaround:
\begin{itemize}
\item replace the direct invocation of the desired Tinker routine (e.g.,
  \texttt{optimize}), with a script of this contents:
\begin{verbatim}
#!/bin/sh
base=`basename $1 .xyz`
out=$base".out"
/path/to/your/tinker/executables/optimize $@ > $out
\end{verbatim}
  For example, you can name this script ``optimize'' and put it in a directory
  that occurs early in your \texttt{PATH}-variable (e.g.,
  \texttt{\$HOME/bin}), before the normal Tinker executables (to be on the
  safe side, remove the Tinker executables from your
  \texttt{PATH}). Alternatively, you can also let \texttt{OGO\_TINKERCMD} point
  to this script.
\item additionally, in the $\ast$.ogo input file, call the Tinker interface with
\begin{verbatim}
LocOptAlgo=tinker:optimize,fileout
\end{verbatim}
\end{itemize}
This (or obvious variations of this recipe for the other Tinker routines) may
get rid of the above error.

The Tinker interface in its current state is not suitable for molecule only
optimizations. This does currently not have any effect on \ogo{}.

\subsection{Interfacing to \textbf{any} other program (package)}
Although \ogo{} provides a numerous set of interfaces to programs implementing
methods ranging from force fields to wavefunction methods, we cannot implement
an interface to every code that we find interesting. Additionally, if one of the
interfaced codes changes input or output structure in a non-compatible manner,
we will need to chase that.

But fear not as there is an easy solution to interface \textbf{any} other code
to \ogo{} with a little bit of simple scripting. This interface can be requested
by setting \texttt{LocOptAlgo=generic}. Additionally, the environment variables
\texttt{OGO\_GENERALCMD} will need to be set to the command to be called for the
actual local optimization. Additionally, if \texttt{OGO\_GENERALOPTS} are set, 
these
options will be appended to the called command.

If this interface is chosen, \ogo{} will create individual directories for each 
local
optimization task. In such directory, a file \texttt{input.xyz} within the 
initial guess
coordinates in xyz-format will be created. Subsequently, the 
\texttt{OGO\_GENERALCMD} will
be executed within the directory (optinally with the options appended) which is 
supposed
to read in the coordinates, run the local optimization and return the results
in an xyz-formatted file \texttt{output.xyz}. In this file, the second line is 
supposed
to contain the fitness of the optimized structure in kJ/mol as the second token.

Happy \ogo{} scripting!

\subsection{Chaining multiple local optimizations}
Any number of local optimizations can be chained. This is easily achieved by
specifying \texttt{LocOptAlgo= chained:} followed by a
\texttt{\textbar}-separated (pipe-separated) list of the local optimization
options (inluding their possible options) to be chained.

\subsection{Build-in local optimizations}
\label{sec:geomlocopt}
A variety of local optimization algorithms are available build-in in \ogo{}. 
They all have in common that they require a backend specification as described 
in the following sections.

Please note that not all of the local optimization algorithms now described are 
classically seen as local optimization but some of them are used in other 
groups as global optimizations engines. Despite the semantics, we are convinced 
that coupling mutliple global optimization engines can be benefical. No bigger 
offense should be taken by naming them local optimizations.

A word on syntax: unless otherwise noted, options to local optimizations are 
semicolon separated. Classic choices for the local optimizations include:

\begin{itemize}
  \item \texttt{chained:XXX|YYY} which chains the local optimizations 
\texttt{XXX} and \texttt{YYY}. Any number of local optimizations can be chained 
and are \textit{pipe} | separated.
  \item \texttt{none:backend=XXX} which uses the backend \texttt{XXX} for 
single-point only evaluations.
  \item \texttt{lbfgs:} to use a L-BFGS optimization through RISO's version of 
the original Nocedal implementation. Recommended choice. Options are:
    \begin{itemize}
      \item \texttt{backend=XXX} uses backend \texttt{XXX} for this 
optimization. Mandatory option.
      \item \texttt{maxiter=XXX} sets the maximal number of iterations for this 
local optimization to \texttt{XXX}. Default: global default. Optional input.
      \item \texttt{lineiter=XXX} sets the number of line search iterations to 
\texttt{XXX}. Default: 50. Optional input.
      \item \texttt{gtol=X.X} sets the gradient tolerance to \texttt{X.X}. 
Default: 0.9. Optional input.
      \item \texttt{nocorrs=X} sets the number of corrections to \texttt{X}. 
Default: 7. Optional input.
      \item \texttt{absconvthresh=X.XX} sets the absolute convergence threshold 
to \texttt{X.XX}. Default: global default. Optional input.
      \item \texttt{tryresets=X} sets the number of resets of the L-BFGS state 
to \texttt{X}. Default: 3. Optional input.
    \end{itemize}
  \item \texttt{apachecg:} to use Apache's conjugate gradient local 
optimization, options include:
    \begin{itemize}
      \item \texttt{backend=XXX} uses backend \texttt{XXX} for this 
optimization. Mandatory option.
      \item \texttt{maxiter=XXX} sets the maximal number of iterations for this 
local optimization to \texttt{XXX}. Default: global default. Optional input.
      \item \texttt{usefletcher=true} enables Fletcher-Reeves updates over 
Polak-Ribiere updates. Default: Polak-Ribiere. Optional input.
      \item \texttt{absconvthresh=X.XX} sets the absolute convergence threshold 
to \texttt{X.XX}. Default: global default. Optional input.
    \end{itemize}
  \item \texttt{flemin:} to use NUMAL's flemin optimization. Options:
    \begin{itemize}
      \item \texttt{backend=XXX} uses backend \texttt{XXX} for this 
optimization. Mandatory option.
      \item \texttt{maxiter=XXX} sets the maximal number of iterations for this 
local optimization to \texttt{XXX}. Default: global default. Optional input.
      \item \texttt{convthresh=X.XX} sets the convergence threshold in the 
fitness to \texttt{X.XX}. Default: global default. Optional input.
      \item \texttt{convthreshgrad=X.XX} sets the convergence threshold in the 
gradient to \texttt{X.XX}. Default: global default. Optional input.
    \end{itemize}
  \item \texttt{rnk1min:} to use NUMAL's rnk1min optimization. Options:
    \begin{itemize}
      \item \texttt{backend=XXX} uses backend \texttt{XXX} for this 
optimization. Mandatory option.
      \item \texttt{maxiter=XXX} sets the maximal number of iterations for this 
local optimization to \texttt{XXX}. Default: global default. Optional input.
      \item \texttt{convthresh=X.XX} sets the convergence threshold in the 
fitness to \texttt{X.XX}. Default: global default. Optional input.
      \item \texttt{maxstep=X.X} sets the maximal step to \texttt{X.XX}. 
Default: 0.1. Optional input.
      \item \texttt{convthreshgrad=X.XX} sets the convergence threshold in the 
gradient to \texttt{X.XX}. Default: global default. Optional input.
    \end{itemize}

\end{itemize}

    \begin{itemize}
      \item 
    \end{itemize}


Please note that the \texttt{flemin}, \texttt{praxis}, \texttt{rnk1min} 
algorithms may only be used if you are in possession of Lau's Java Numal 
book\cite{javanumalbook} (and 
therefore the accompanying source code and jar-file). In this case, you can add 
the numal.jar to the classpath prior 
to \ogo{} excecution and make use of the optimization algorithms implemented in 
Numal.

%TODO localheat not included with source yet!

% One of the global optimization algorithms \ogo{} included are local heat 
% pulses 
% as originally developed by Arnulf Moebius. The version in \ogo{} is fairly 
% customized
% in comparison to the original. You can specify this using: 
% \texttt{localheat:} followed by the actual local optimization (may be any of 
% the ones available in \ogo{} and the options for the local heat pulses, 
% separated by ; so \texttt{localheat:lbfgs:lennardjones;OPTIONS}. The options 
% are in turn a comma separated list of a variety of options. Please note that 
% whilst you must have the correct number of commas, you may leave the space in 
% between them empty and choose the default. We choose this input format to 
% ensure that users are well aware that this is not a black-box method. The 
% options are as follows: 
% \texttt{MODE,doCDDD,eqIter,iters,moveCOMs,scaleFac,startAmpl,temp,useMetro,
% startEuler,sigmaX,sigmaY,sigmaZ}.
% \begin{itemize}
%  \item mode: the mode for the local heat pulses. We support the following 
% choices: 0 (default) move 5 atoms/COMs, 1: move 10\% of the atoms/COMs, 2: 
% move 
% 0 to 5 atoms/COMs, 4: return 0 to 10\% of the atoms/COMs, 10: shoot a pulse
% into 
% the cluster and disturb within a 3D Gaussian, 11: shoot a pulse on the 
% cluster 
% surface and disturb within a 3D Gaussian.
% \item doCDDD: if we should check the geometries sanity after perturbation 
% using
% CD/DD. Default: false.
% \item eqIters: the number of equillibration iterations before cooling the
% amplitudes. Default: 3.
% \item iters: the number of iterations of the local heat pulses. Default: 10.
% \item moveCOMs: if the COMs of the building blocks or the atoms should be
% perturbed. Default: true (move COMs).
% \item scaleFac: how we scale amplitudes, temperature and euler amplitudes.
% Default: 0.9.
% \item startAmpl: what the start amplitude for atom/COM perturbation should be.
% Default: 2\AA{} in $a_0$.
% \item temp: if we use a Metropolis criterion, what the start temperature (in
% Kelvin) should be. Default: 50.
% \item useMetro: if we use the Metropolis criterion instead of the simplistic
% criterion. Default: true (use Metropolis).
% \item startEuler: the initial amplitude for Euler angle perturbation. Default:
% 0.5.
% \item sigmaX: the x-width of the 3D Gaussian (if mode 10 or 11). Default: 0.1.
% \item sigmaY: the y-width of the 3D Gaussian (if mode 10 or 11). Default: 0.1.
% \item sigmaZ: the z-width of the 3D Gaussian (if mode 10 or 11). Default: 0.1.
% \end{itemize}


\section{Internal and external backends}
As discussed above, all internal local optimizations require a backend 
definition. This backend definition can be
either directly inlined or pre-defined and tagged. Please note that we prefer 
the pre-defined version discussed
here over direct inlining.

In order to define a backend a \texttt{CLUSTERBACKEND} block must be in the 
input. Options include:
\begin{itemize}
 \item \texttt{BackendTag=XYZ} to name this backend defintion \texttt{XYZ}. Tag 
must be unique. Mandatory option.
 \item \texttt{Backend=XXX} to provide the fully qualified backend definition. 
Mandatory option.
\end{itemize}

As an example:
\begin{verbatim}
<CLUSTERBACKEND>
BackendTag=lj
Backend=xyz:lennardjones
</CLUSTERBACKEND>
\end{verbatim}
would be used to assign a Cartesian-based LJ force-field the tag \texttt{lj}. 
Subsequently, the tag \texttt{lj}
could be used to refer to this backend in the local optimization definitions.

Actual backend definitions will be covered in the next sections.

\subsubsection{Internal LJ force field}
Very simple, based on standard parameters for \emph{homogenous} LJ clusters of
noble gases (excluding radon). Again, this does \emph{not} include any mixing 
and can therefore only be used for
homogenous clusters. Specify \texttt{lennardjones} as the backend.

\subsubsection{Internal mixed LJ force field}
Also simple, based on standard parameters for \emph{heterogenous} LJ clusters of
noble gases. Through the
use of Lorentz-Berthelot mixing rules, any mixed cluster of noble gases
(excluding radon) can be studied. Specify \texttt{mixedlj} as the backend.

\subsubsection{scaTTM3F}
The \ogo{} contains a Scala-based implementation of the TTM3F force field by
Xantheas \emph{et al.}. Despite being significantly faster than the reference
implementation, this force field is equivalent in behaviour to said original. To
use it, simply specify \texttt{scattm3f:X,Y.Y}, where \texttt{X} is the number
of water molecules your system contains (our implementation also pre-allocates 
some
scratch arrays) and \texttt{Y.Y} is an energy cutoff in $E_h$. This cutoff is of
importance, as (like the reference implementation) the induced dipols may
converge to unphysical solutions which can be very low in energy. It is
recommended to scale the cutoff with system size.

Please also note that the implementation assumes your cluster to only contain
water molecules with internal atom order O/H/H.

\subsubsection{(EVB-)QMDFF}
Thanks Stefan Grimme, we include his QMD force field natively in \ogo{}. 
Additionally,
we support the EVB-QMDFF extension by Bernd Hartke and Stefan Grimme. Before 
running \ogo{}, a force field parameter must be created file with the 
official tools from Stefan Grimme. Subsequently, QMDFF can be configured as
\texttt{qmdff:} followed by a , (comma) separated list of options:
\begin{itemize}
 \item \texttt{params=XXX} sets the path to the parameter file to \texttt{XXX}. Default: \texttt{solvent}.
 \item \texttt{addterm=XXX} add an additional term to QMDFF. Currently only supported: \texttt{electrostatic:PATHTOCHARGEFILE}. Default: no additional term.
\end{itemize}

To configure EVB-QMDFF, specify \texttt{evb-qmdff:} followed by any of the following options:
% TODO specify
\begin{itemize}
 \item \texttt{evb=}
 \item \texttt{qmdffs=}
\end{itemize}

Please note that we are in the testing phase for both EVB-QMDFF and QMDFF for structure global optimization and cannot make any guarantees as to the physical sensibility of obtained results!

\subsubsection{Adaptive choices}
Adaptive choices are in more detail described in section \ref{parameterfit}
\emph{Global parametrization of potentials}.

\section{Global optimization algorithms}
\label{sec:geomglobopt}
All global optimization algorithms are chosen through the \texttt{GlobOptAlgo=}
keyword. The grammar is as follows: 
\texttt{cluster$\{$xover(GEOMXOVERALGO:OPTS)mutation(GEOMMUTATIONALGO:OPTS)$\}
$molecules$\{$ xover(MOLXOVERALGO:OPTS)mutation(MOLMUTATIONALGO:OPTS)$\}$}.
The cluster algorithm specification is required, the molecule specification 
(for explicit degrees of freedom) is optional. Algorithmic choices must end 
with a colon, options are optional and comma-separated.

Valid choices for \texttt{GEOMXOVERALGO} are:
\begin{itemize}
  \item \texttt{germany:} a very basic one-point genotype crossover. Options:
  \begin{itemize}
    \item \texttt{gausswidth=X.X} sets the width of the Gaussian deciding for 
the cutting spot to X.X, default 0.3.
  \end{itemize}
  \item \texttt{portugal:} a $N$-point genotype crossover.
  \begin{itemize}
    \item \texttt{nocuts=X} sets the number of cuts to \texttt{X}. Default: 1.
  \end{itemize}
  \item \texttt{sweden:} our $N$-species phenotype crossover.
  \begin{itemize}
    \item \texttt{cutstyle=} specifies the cutting style, valid choices 0, 1 or 
2. 0: cutting plane is always through the total COM of the cluster. 1: 
Gauss-distributed cutting plane. 2: normal-distributed cutting plane. Default: 
2.
  \end{itemize}
  \item \texttt{lapland:} a phenotype crossover with implicit exchange.
  \begin{itemize}
    \item \texttt{cutstyle=}, valid choices 0, 1 or 2. Default: 2.
  \end{itemize}
  \item \texttt{iceland:} a merging phenotype operator.
  \begin{itemize}
    \item \texttt{cutstyle=X}, valid choices 0, 1 or 2. Default: 2.
    \item \texttt{colldetect=X} Choses a collision detection engine. See global 
option for valid choices. Default: global setting.
    \item \texttt{blowcoll=X.X} specifies the collision blow factor to be 
\texttt{X.X}. Default: global setting.
    \item \texttt{blowdiss=X.X} specifies the dissociation blow factor to be 
\texttt{X.X}. Default: global setting.
    \item \texttt{inittrust=X.X} specifies the initial trust radius for the 
merge. Default: 0.1.
    \item \texttt{stoptrust=X.X} specifies the stopping trust radius for the 
merge. Default: 1E-5.
    \item \texttt{mergeiter=X} specifies the number of iteration for the merge. 
Default: 100.
    \item \texttt{nointerpoints=X} specifies the number of interpolation points 
for the merging. Default: 5.
    \item \texttt{optbounds=} specifies the optimization bounds for the merging 
as a $/$-separated array. Order: lower translation bound, higher translation 
bound, lower rotation bound, higher rotation bound. Defaults: 
\texttt{0.0/15.0/0.0/2*pi}
  \end{itemize}
  \item \texttt{aland:} a fitness-based phenotype operator with optional 
merging.
  \begin{itemize}
    \item \texttt{cutstyle=X}, valid choices 0, 1 or 2. Default: 2.
    \item \texttt{mergingpheno}, enables merging. Default: off.
    \item \texttt{colldetect=X} Choses a collision detection engine. See global 
option for valid choices. Default: global setting.
    \item \texttt{blowcoll=X.X} specifies the collision blow factor to be 
\texttt{X.X}. Default: global setting.
    \item \texttt{blowdiss=X.X} specifies the dissociation blow factor to be 
\texttt{X.X}. Default: global setting.
    \item \texttt{inittrust=X.X} specifies the initial trust radius for the 
merge. Default: 0.1.
    \item \texttt{stoptrust=X.X} specifies the stopping trust radius for the 
merge. Default: 1E-5.
    \item \texttt{mergeiter=X} specifies the number of iteration for the merge. 
Default: 100.
    \item \texttt{nointerpoints=X} specifies the number of interpolation points 
for the merging. Default: 5.
    \item \texttt{optbounds=} specifies the optimization bounds for the merging 
as a $/$-separated array. Order: lower translation bound, higher translation 
bound, lower rotation bound, higher rotation bound. Defaults: 
\texttt{0.0/15.0/0.0/2*pi}
  \end{itemize}
  \item \texttt{snaefellsjoekull:}
  \begin{itemize}
    \item \texttt{cutstyle=X} specifies a cut style. Allowed: 0 (full random), 
1 (Gauss-distributed), 2 (inverted Gauss-distributed). Default: 0.
    \item \texttt{gausswidth=X.X} the standard deviation if a Gauss-cut is 
chosen. Default: 1.0.
    \item \texttt{adjustradius=true/false} if the radius of the spheres should 
be adjusted. Default: false.
  \end{itemize}
  \item \texttt{noxover:} disables crossovers in the global optimization.
  \item \texttt{mutationasxover:GEOMMUTATIONALGO:OPTS} uses the mutation 
specified by \texttt{GEOMMUTATIONALGO} as a crossover.
  \item \texttt{multiple:XX\%GEOMXOVER1:OPTS|YY\%GEOMXOVER2:OPTS...} couples 
multiple crossover operators together. Each definition starts with the percent 
probabilty for this operator to be used. Different operators are separated by 
pipes |. The percentages must add up to 100\%.
  \item \texttt{chained:XX\%GEOMXOVER1:OPTS|YY\%GEOMXOVER2:OPTS...} chains 
multiple crossover operators together. Each definition starts with the percent 
probabilty for this operator to be used in the chain. Different operators are 
separated by pipes |. Each percentage is independent.
\end{itemize}

We do recommend to choose \texttt{sweden} if one is not certain what the best
crossover is.

Valid choices for \texttt{GEOMMUTATIONALGO} are:
\begin{itemize}
  \item \texttt{germany:} for a basic genotype mutation of the COMs and 
rotations for the molecules.
  \begin{itemize}
    \item \texttt{lowcom=X.X} the minimal translation for the COMs. Default: 
0.5 bohr.
    \item \texttt{highcom=X.X} the maximal translation for the COMs. Default: 
1.5 bohr.
    \item \texttt{widthCOM=X.X} the width of the Gaussian centered between the 
two. Default: 0.1.
    \item \texttt{loweuler=X.X} the minimal rotation of the COMs. Default: 0.0 
(in the specific Euler angles definition.
    \item \texttt{higheuler=X.X} the maximal rotation of the COM. Default: 1.0 
(in the specific Euler angles definition.
    \item \texttt{widtheuler=X.X} the width of the Gaussian centered between 
the two Euler borders. Default: 0.1.
  \end{itemize}
  \item \texttt{montecarlo:} for a Monte-Carlo based mutation in the atomic 
space.
  \begin{itemize}
    \item \texttt{mode=} to adjust the move mode. Valid choices: \texttt{all}, 
\texttt{some}, \texttt{one}.
    \item \texttt{maxmove=X} specifies the maximum move. Default: 0.2 (in bohr).
  \end{itemize}
  \item \texttt{finland:} for a one-parent phenotype crossover.
  \begin{itemize}
    \item \texttt{cutstyle=X}, valid choices 0, 1, 2. Default: 2.
    \item \texttt{blowcoll=X.X} to set the blow factor for the collision 
detection. Default: global setting.
    \item \texttt{blowdiss=X.X} to set the blow factor for the dissociation 
detection. Default: global setting.
  \end{itemize}
  \item \texttt{norway:} for a one-parent packing operator.
  \begin{itemize}
    \item \texttt{colldetect=} to set the collision detection algorithm. 
Default: global setting.
    \item \texttt{dissdetect=} to set the dissociation detection algorithm. 
Default: global setting.
    \item \texttt{blowcoll=X.X} to set the blow factor for the collision 
detection. Default: global setting.
    \item \texttt{blowdiss=X.X} to set the blow factor for the dissociation 
detection. Default: global setting.
  \end{itemize}
  \item \texttt{xchangemut:} for an explicit exchange mutation.
  \begin{itemize}
    \item \texttt{mode=} the exchange mode. Valid choices: \texttt{single}, 
\texttt{multiple}. Default: \texttt{single}.
    \item \texttt{gausswidth=X.X} if multiple is chosen, the width of the 
Gaussian used to determine how many should be exchanged. Default: 0.4.
  \end{itemize}
  \item \texttt{graphbasedmut:} the original GDM doing a graph-based analysis 
of the cluster to determine the least connected building block and move it to a 
better position.
  \begin{itemize}
    \item 
  \end{itemize}
\end{itemize}

To be continued... %TODO


\subsection{HACTAR}
\label{sec:geomhactar}
HACTAR is...... \textbf{not yet ready for prime time. Please do not use!} %TODO

Input syntax: 
\texttt{hactar$\{$HACTAROPTIONS$\}$cluster$\{$xover()mutation()$\}$}.  The 
cluster specifications for both \texttt{xover()} and \texttt{mutation()} can 
(and should) contain more than one algorithmic definition. These definitions 
follow the \texttt{multiple:} syntax explained above, without the leading 
\texttt{multiple:} keyword.

Currently no molecular 
HACTAR exists. \texttt{HACTAROPTION} are a series of optional, comma-separated 
options. The following input keywords exists:
\begin{itemize}
  \item \texttt{mode=} Wich moving mode to use. Possible modes are 
\texttt{all}, \texttt{some}, \texttt{one}. Default: \texttt{some}.
  \item \texttt{dometropolis} To enable the Metropolis criterion. Default: 
accept steps that lower the energy, no Metropolis criterion.
  \item \texttt{beta=X.X} If the Metropolis criterion is enabled, which beta 
factor to use. Default: 1.0.
  \item \texttt{maxtoreset=X} After how many unsuccessful operations a reset to 
the initial algorithmic configuration should take place. Default: 100.
  \item \texttt{stepstopring=X} After how many steps we print the current 
algorithmic configuration to standard out. Default: 1000.
  \item \texttt{maxpercstep=X.X} What is the maximum percentage the 
configuration is allowed to be moved in one step. Default: 1\%.
  \item \texttt{dopercentages} evaluate the improvements as a percentage, not
as the absolute.
\end{itemize}


\section{Diversity checkers}
In order to maintain a sufficient diversity in the genetic pool, we use so 
called
diversity checkers. Their quality (and performance) are crucial for the success 
of
your global optimization.

The default is a fitness-based diversity criterion with a threshold of 
$1\cdot10^{-6}\:E_h$. This threshold may be too tight for your optimization and 
is related to the convergence threshold of the local optimization. To change 
the default, the \texttt{DiversityCheck=} keyword must be used. Options are:
\begin{itemize}
  \item \texttt{fitnessbased:X.X}\\
    which is the simple fitness-based diversity checker with a threshold of 
\texttt{X.X}.
  \item \texttt{percfitnessbased:X.X}\\
   which is the simple fitness diversity checker based on the percentage 
difference w.r.t. the individual with the smaller fitness. \texttt{X.X} is the 
percentage in the interval [0,100].
  \item \texttt{hundtoverlap:CONFIG}\\
    to use Hundt's overlap check as a diversity criterion using the 
configuration options in \texttt{CONFIG}, semicolon separated.
  \item \texttt{fitnesshundtoverlap:X.X,CONFIG}\\
    to use a fitness-prescreened version (with threshold \texttt{X.X}) of 
Hundt's overlap check as a diversity criterion using the configuration options 
in \texttt{CONFIG}, semicolon separated.
\end{itemize}


\section{Running the job}
After calling, e.g.,
\begin{verbatim}
java -jar ogolem.jar --shmem MYJOB.ogo 2
\end{verbatim}
there will, potentially, appear (a lot of files) in the directory the job was 
started from. Additionally, depending upon the number of threads allowed, a 
java process will start eating up CPU time. Concerning the latter: it is not 
recommended to use all CPUs on a workstation which is supposed to be still used 
for anything else.

\ogo{} will clean up all automagically generated input files and/or directories
for other programs in case of a sucessfull call to that program. In case that
something unexpected (e.g. failing convergence) happens, the files will be left
for inspection.

Interesting datafiles are the folder named like your input file, where
structures and output (the latter one is to be inspected to see whether all
input keywords are correctly read in) are collected. Additionally, there is a
log file containing timestamps and informations whenever a new individual took
the first rank in the pool.

Once the run is complete, there will be a lot of files of form
\texttt{rankNUMBERgeometryNUMBER.xyz} appearing in that folder. This is the
final pool of structures, where the first number is the rank in the pool (best
candidate has rank 0) and the second is the ID of the geometry to give a measure
when it was found in the run.

\ogo{} might throw warnings or errors. These will be written to System.out and
System.err and are of interest.

\section{Cluster analysis}
\ogo{} provides some, admittingly very trivial, means to analyze the outcome of
a global optimization of clusters and, perhaps more important, to check a
running job using the written restart files. For this, you call the
\texttt{ogolem.jar} with the
\begin{verbatim}
--clusters
\end{verbatim}
option.

The analysis program needs a binary pool, either the final \texttt{pool.bin}
or a temporary \texttt{IntermediateClusterPool.bin}. This is specified using the
\texttt{-i} flag followed by whitespace, e.g.
\begin{verbatim}
java -jar ogolem.jar --clusters -i IntermediateClusterPool.bin
\end{verbatim}

Some other options are possible
\begin{itemize}
  \item \texttt{-binstructs}\\
	bin structures by number of structures instead of energies (which is
the default)
  \item \texttt{-dissblow}\\
	the blow factor for the dissociation detection, defaults to 1.8.
  \item \texttt{-getstructs}\\
	 create a directory named \texttt{structs} and write the structures of
this pool there in the xyz-file format. Good option for intermediate checks.
  \item \texttt{-ljstrains}\\
	calculate strain energies for LJ clusters. Defaults to off.
  \item \texttt{-nocomdiffs}\\
	disable analysis of COM differences (on by default).
  \item \texttt{-nodissdetect}\\
	disable dissociation detection (on by default).
  \item \texttt{-noenergies}\\
	disable scanning of energies (on by default).
  \item \texttt{-noinertias}\\
	disable moments of inertia analysis (on by default).
  \item \texttt{-noofbins}\\
	the number of bins for structure binning. Defaults to 100.
  \item \texttt{-threshdiff}\\
	threshold for moments of inertia, defining when they are considered to
be different. Defaults to 2.5E-1.
  \item \texttt{-threshequal}\\
	threshold for moments of inertia, defining when they are considered to
be same. Defaults to 1E-1.
\end{itemize}

We want to encourage you here to contribute to the \ogo{} by proposing
cluster analysis techniques that are of interest to you!


\chapter{Global parametrization of potentials, pseudopotentials, ...}

This part of \ogo{} started as a way to globally optimize potentials in
a system-specific fashion to later use them for cluster structure optimization.
It has since significantly evolved from that basis towards optimizing
potentials, pseudopotentials, and test benchmark functions. However, some traces
of its ancestry are still present, i.e., in the names of keywords.

\textbf{IMPORTANT:\\
This part of the manual is not complete yet and does, e.g., not cover all
possible adaptivable choices! It also is outdated as the input syntax has been 
adjusted.}

The part for the global optimization of potentials in the \ogo{} was primarily
designed with the idea in mind to allow for a system-specific parametrization of
potentials against a reference for later usage in the geometry structure
optimization part. Therefore, it is strongly coupled to that part.

Of importance is the definition of the fitness in this part. In the easiest case
(no weights, no penalty functions, exact computation), it is
\begin{equation}
F=\sum_{i=0}^{N}abs\left( E_{i,fit}-E_{i,ref} \right)
\end{equation}
where $N$ is the number of reference points with their reference energies
$E_{i,ref}$ and the fitted energies $E_{i,fit}$. Therefore, an exact agreement
between reference and fitted energies causes the fitness to vanish, making 0.0
the optimal one.

\label{parameterfit}
\section{General input structure}
Due to the tight interaction between this part and the geometry optimization
part, all direct input for the parameter optimization resides in the
\texttt{.ogo} file. Therefore, there \emph{needs} to be a geometry definition.
This is useful if one wants to use the feature of automatical generated
reference structures and is otherwise only a small nuissance.

All direct input is encapsulated into \texttt{<ADAPTIVE>} tags, e.g.,
\begin{verbatim}
###OGOLEM###
<GEOMETRY>
NumberOfParticles=2
<MOLECULE>
Ar;0.0;0.0;0.0
</MOLECULE>
<MOLECULE>
Ar;0.0;0.0;0.0
</MOLECULE>
</GEOMETRY>

<ADAPTIVE>
AcceptableFitness=1E-8
<REFERENCE>
<PATH>
arar_2.xyz
</PATH>
<ENERGY>
1E-5
</ENERGY>
</REFERENCE>
<REFERENCE>
<PATH>
arar_22.xyz
</PATH>
<ENERGY>
1.2E-5
</ENERGY>
</REFERENCE>
</ADAPTIVE>
\end{verbatim}

If one does not use the by default disabled feature of automatic reference point
generation, the geometry definition does not need to be identical to the
reference points. Also, in general the reference points may (depending on the
adaptivable chosen) have different compositions and number of atoms.

Additionally, there exists an optional file specific to the parametrization
part. This file defines lower and upper boundaries to each parameter and will be
used for this purpose if present, otherwise default borders will be used which
might be very inefficient for your specific purpose.

The file should be named \texttt{yourinputname-borders.aux}, where
\texttt{yourinputname} is the prefix of the used \texttt{.ogo} file, and the
general format is
\begin{verbatim}
###OGOLEMAUX###
-3.0        2.0
0.1         0.2
1E3         1E5
-1E3        -1E1
\end{verbatim}
where for each individual parameter a lower boundary is assigned on the left and
a upper one on the right side. These bounds typically (exceptions are mentioned
in the corresponding paragraph of the adaptivable choice) are in atomic units.


It is important to note that to-date the length of the specified array of
borders needs to agree with the overall number of parameters for the fit and the
order needs to agree with the internal representation. Depending on the actual
interface chosen, it may therefore by reasonable to start \ogo{} with default
borders to figure these things out.

The automatic calculation, how many parameters and which parameters are needed
for a given set of reference points may be too conservative and generate too
many parameters. This of course highers the problem dimensionality and makes the
genetic operations and local optimizations be less efficient. If this is the
case for you, you may provide a file named \texttt{yourinputname-stub.aux},
where \texttt{yourinputname} is the prefix of the used \texttt{.ogo} file. This
file will then be used as a reference stub for the global optimization. Please
be aware that any errors here can cause really nasty behaviour of the
optimization. In general, the format of this file should follow the later
described parameter output format. The numerical values of the parameters given
can be any floating point number and do not have any effect on the optimization.
Again, this potentially very harmful option should only be used if one knows
what one is doing!

\section{Input of reference points}
The reference points are handed over through input of form
\begin{verbatim}
<REFERENCE>
<PATH>
pathToCartesianReference.xyz
</PATH>
<ENERGY>
0.1
</ENERGY>
<REFCHARGES>
0;1;2
</REFCHARGES>
<REFSPINS>
0;2;1
</REFSPINS>
<REFBONDS>
bonds.list
</REFBONDS>
<REFWEIGHT>
2.4
</REFWEIGHT>
<REFMAXDIFF>
1E-5
</REFMAXDIFF>
</REFERENCE>
\end{verbatim}
All adaptive backends support energy fits currently (this may change in the future) and for these \texttt{<PATH>} and \texttt{<ENERGY>} tags are mandatory and hold the
path to the Cartesian structure and its energy (in hartree) of this reference
point. The other tags are optional with both \texttt{<REFSPINS>} and
\texttt{<REFCHARGES>} following the exact same input syntax as described above
for the geometry structure optimization.

Please note that the \texttt{<PATH>} tags must always be located before any 
other tags!

The \texttt{<REFBONDS>} tags may contain the path to either a list or a full
matrix of the bonds in this reference geometry. If the file suffix is
\texttt{.list}, a list will be assumed, otherwise a full matrix needs to be in
the file (in most cases, one wants to use the list option). If these tags are
not given, \ogo{} makes estimates based on a distance criterion. The list needs
to be in format
\begin{verbatim}
0 1
6 5
2 3
4 5
\end{verbatim}
where each row specifies the two atoms (once more counting starts at 0) between
a bond exists. Empty lines are permitted, no order needs to be pertained. If a
full matrix needs to be read in, the file should contain
\begin{verbatim}
true true true
true true false
true false true
\end{verbatim}
for (in this case) a water molecule. No empty lines are permitted.

The \texttt{<REFWEIGHT>} tag assigns a
weight to this specific point, a feature that might be usable if e.g. the
description of a certain region is important. The \texttt{<REFMAXDIFF>} tag
defines how much the fitted energy is allowed to differ from the reference
before a penalty is applied. This last tag obviously must be used with a penalty
function enabled separatly.

There is growing support for other properties. Please check if your problem supports the following. If not, patches are welcome!

\subsection{Parametrizing energy order}
Energy order refers to the ordering of different atomic arrangements (e.g., crystal structures). It can be input as follows:
\begin{verbatim}
<REFERENCE>
<ENERGYORDER>
ConsiderRelativeEnergies
First;pathtofirst.xyz;-945.64
Second;pathtosecond.xyz;-1213.3
</ENERGYORDER>
</REFERENCE>
\end{verbatim}
where \texttt{ConsiderRelativeEnergies} is an optional tag (must be in first line if used) to not use total energies for the parametrization
but only relative ones. Followed by arrangements having a tag as the first element, a semicolon as a separator, the path (relative or absolute)
to the atomic positions in either xyz or z-matrix format, another semicolon separator, and the energy (typically in Hartree, can be relative if above relative energies were requested).

\subsection{Parametrizing bulk modulus}
Bulk moduli are of importance for condensed matter applications. They can be input as follows:
\begin{verbatim}
<REFERENCE>
<BULKMODULUS>
Li(bcc):150.624
</BULKMODULUS>
</REFERENCE>
\end{verbatim}
where \texttt{Li(bcc)} refers to the bcc phase of lithium, followed by colon and the bulk modulus (typically in atomic units).

\subsection{Parametrizing equilibrium volume}
Equilibrium cell volumes are of importance for condensed matter applications. They can be input as follows:
\begin{verbatim}
<REFERENCE>
<CELLVOLUME>
Li(bcc):42.4242
</CELLVOLUME>
</REFERENCE>
\end{verbatim}
where \texttt{Li(bcc)} refers to the bcc phase of lithium, followed by colon and the equilibrium cell volume (typically in atomic units).

\subsection{Parametrizing nuclear forces}
Nuclear forces can be input as follows:
\begin{verbatim}
<REFERENCE>
<PATH>
...
</PATH>
<FORCES>
ForcesFile=pathtoforces.dat
</FORCES>
</REFERENCE>
\end{verbatim}
where \texttt{ForcesFile=} points to a file (relative or absolute path) containing the forces for the current reference in a Cartesian xyz format. The forces input requires also a \texttt{<PATH>} environment with atomic positions as defined above.

\subsection{Parametrizing $\delta$ gauge}
$\delta$ gauges are of importance for condensed matter applications. They can be input as follows:
\begin{verbatim}
<REFERENCE>
<PATH>
...
</PATH>
<DELTAGAUGE>
Li(bcc):0.0
</DELTAGAUGE>
</REFERENCE>
\end{verbatim}
where \texttt{Li(bcc)} refers to the bcc phase of lithium, followed by colon and the $\delta$ gauge (typically 0.0 in atomic units). Currently, the $\delta$ gauge requires also a \texttt{<PATH>} environment with atomic positions as defined above.

\subsection{Parametrizing electron density}
Electron density (on a grid) can be input as follows:
\begin{verbatim}
<REFERENCE>
<PATH>
...
</PATH>
<DENSITY>
DensityFile=pathtodensity.dat
ReferenceTag=myTag
</DENSITY>
</REFERENCE>
\end{verbatim}
where \texttt{DensityFile=} points to a file (relative or absolute path) containing the density on a grid for the current reference. First line contains the Cartesian grid dimensions (whitespace separated) and subsequent lines contain one data point on that grid. The electron density input requires also a \texttt{<PATH>} environment with atomic positions as defined above. The second line of the \texttt{<REFERENCE>} environment must define a tag for this density using \texttt{ReferenceTag=}.

\subsection{Mixing properties for the same reference}
Some of the above properties can be mixed if the same reference is meant (the definition of that is somewhat backend dependent!). For this, simply add multiple property tags (e.g., \texttt{<CELLVOLUME>} \texttt{BULKMODULUS} in the same \texttt{<REFERENCE>} environment.

\section{Input keywords}
Again, there are a lot of tunables to this part of \ogo{} with some of them more
important than others. It is important to note that a certain setting might be
optimal for one fit but not for another.

\begin{itemize}
   \item \texttt{AcceptableFitness=}\\
        \label{page2:acceptableFitness}
	which fitness is considered to be acceptable and causes the optimization
to stop. Defaults to 0.0 which is by our definition total agreement between all
references and the fit.
  \item \texttt{AdaptivableChoice=}\\
	what to adapt/parametrize. See details below.
  \item \texttt{AllRefsSameChargesAndSpins=}\\
	set to true if all reference geometries have the same charge and spin. Use
	with extraordinary care only!
  \item \texttt{AnyParamHistory=}\\
	set to true if the genetic history of parameters should be tracked. This
	implies a higher use of memory.
  \item \texttt{DoNiching=}\\
  \item \texttt{DoubleMorseDMax=}\\
	if the double morse adaptivable is used, the $d_{max}$ value. Deprecated!
	set to true if niching should be enabled.
  \item \texttt{NichesPerDim=}\\
	if niching is enabled, the number of niches per (static) grid dimension.
  \item \texttt{MaxIndividualsPerNiche=}\\
	if niching is enabled, the maximum number of individuals per niche.
  \item \texttt{MaxNonConvSteps=}\\
	maximum number of steps w/o fitness progress in the internal local
optimization. Defaults to 100.
  \item \texttt{MaxTasksToSubmit=}\\
	maximum number of tasks that are submitted to the threadpool at once.
Defaults to 1000000. Please keep this setting if you are not absolutely certain
that you benefit from a change.
  \item \texttt{ParamBorderPrint=}\\
	set to false if the parameter borders should be printed.
  \item \texttt{ParameterDetailedStats=}\\
	set to true if detailed statistics for the global optimization are wanted.
  \item \texttt{ParamEnergyDiv=}\\
	the minimal fitness diversity needing to exist between two individuals
in the pool. Defaults to 1E-8 which might be (depending on e.g. the convergence
criterion of the local optimization) too small and cause premature convergence
of the pool.
  \item \texttt{ParamGlobOptIter=}\\
	the number of global optimization steps. Defaults to 9900. Must be
	increased for production runs!
  \item \texttt{ParamLinesearchEnergyDecr=}\\
	threshold for the fitness decrease in the linesearch. Defaults to 1E-10.
  \item \texttt{ParameterGlobOpt=}\\
	which global optimization ingredients to use. Syntax is
	\texttt{parameters\{xover()mutation()\}}. Choices include: \texttt{portugal:}
	with optional \texttt{nocuts=N} argument for an N-point genotype xover.
	and \texttt{germany:} for both mutation and xover. Details below.
  \item \texttt{ParamLocOptIter=}\\
	maximum number of iterations for the parameter local optimization.
Defaults to 5000.
  \item \texttt{ParameterLocOpt=}\\
	which local optimization engine should be used for the local optimization.
	Choices include: apachecg (for Apache's conjugate gradient implementation),
	lbfgs (for an L-BFGS), and bobyqa (for Powell's BOBYQA).
  \item \texttt{ParamLocOptMaxStep=}\\
	maximum stepsize in the local optimization. Defaults to 1E-5.
  \item \texttt{ParamSeedingFolder=}\\
	specify the folder where seeds for a seeding initialization are located.
  \item \texttt{ParamSerializeAfterNewBest=}\\
	set to false if the genetic pool should not be stored to disk after each
	new best individual found.
  \item \texttt{ParamThreshDiv=}\\
	the minimal difference between the same gene of two individuals. Applies
only if the corresponding diversity check is enabled. Defaults to 1E-8 which
might be (depending on e.g. the convergence
criterion of the local optimization) too small and cause premature convergence
of the pool.
  \item \texttt{ParamThreshLocOptGradient=}\\
	threshold for the gradient convergence. Defaults to 1E-8.
  \item \texttt{ParamThreshLocOptParam=}\\
	threshold for the parameter convergence. Defaults to 1E-8.
  \item \texttt{ParamsToBorders=}\\
	whether the parameters should be put back into their borders. Otherwise,
a local optimization may cause the parameters to run out of the borders.
Defaults to false.
  \item \texttt{ParamsToSerial=}\\
	After how may steps a binary intermediate parameter pool is written to
disk. Defaults to 10000;
  \item \texttt{PercentageForDiffs=}\\
	if percentage-scaled differences are enabled, what percentage should be
	used. Default: 1 [in percent].
  \item \texttt{PopulationSize=}\\
	the pool size for the parameter fit. Defaults to 100.
  \item \texttt{UsePercentageScaledDiffs=}\\
	set to true to enable percentage-scaled differences for the maximum allowed
	difference.
\item \texttt{WhichNicher=}\\
	selects the niching protocol being used. First an integer selection
	for the niching algorithm, followed by a semicolon and the nicher. Choices
	are 0 for static grid niches without ratio consideration and 1 for static
	grid niches with ratio consideration.
  \item \texttt{WhichDiversityCheck=}\\
	integer choice which diversity check is to be used. Defaults to 0, which
is a fitness based diversity check. Also possible: 2, which compares the
individuals based on their gene values.



% \item \texttt{ExactFitnessComp=}\\
%	whether the fitness is computed based on the relative differences to the
%first reference point or the absolute reference energies. Defaults to true,
%which uses the absolute energies.
%keyword defines how many should be generated. Defaults to 100.
%  \item \texttt{ParamLocOptIter=}\\
%	maximum number of iterations for the parameter local optimization.
%Defaults to 5000.
%  \item \texttt{PenaltyConst=}\\
%	the $k$ for the penalty function of form $k\cdot x^y$. Defaults to 10.0
%(please pay attention that the penalty is in hartree).
%  \item \texttt{PenaltyPow=}\\
%	the $y$ for the penalty function of form $k\cdot x^y$. Defaults to 2.0
%(please pay attention that the penalty is in hartree).
%  \item \texttt{UseMaxAllowedDiffs=}\\
%	enables the usage of a penalty function if the difference at a reference
%point between reference and fit is bigger than a certain threshold. Defaults to
%false (off).
%  \item \texttt{ReferenceEnergiesToFirst=}\\
%	whether or not the reference energies should just be computed as a
%difference to the first one. Defaults to false.
  
\end{itemize}

\section{Adaptivable choices}
\ogo{} supports a to-date only a few adaptivable choices. If there should be a
(preferably free) program that you want to see supported and/or the methods
listed here to do not satisfy you, feel free to write a mail to
dieterich@ogolem.org . It is necessary to provide reference in- and output to
the program you want to see support for and (in case that it is no freeware) at
least a binary of the program/engine is required.

In general, all adaptivable choices are easily called through input of form
\begin{verbatim}
AdaptivableChoice=IDENTIFIER:FLAGS
\end{verbatim}

\subsection{Adaptive GUPTA potential}
The adaptive GUPTA potential can be called using
\texttt{adaptivegupta:DISTBLOW,ENABLECACHE}, where \texttt{DISTBLOW} is a blow
factor defining when to neglect contributions. \texttt{ENABLECACHE} is a boolean
defining whether or not to enable the internal caching procedures. The later
should only be used when all geometries are of precisely the same system. In
general, it can and should be anabled for cluster global optimization and may be
enable for parameter global optimization if (and \emph{only} if) all reference
structures are of the same system. Using caching provides a significant speedup
in comparision to the uncached procedures.

The used function definition is as follows:
\begin{equation}
E(a,b)=A(a,b)\cdot
\exp{\left[-p(a,b)\left(\dfrac{r_{ab}}{r_0(a,b)}-1\right)\right]}
-\sqrt{\chi(a,b)^2\cdot\exp{\left[-2\cdot q(a,b)
\left(\dfrac{r_{ab}}{r_0(a,b)}-1\right)\right]}}
\end{equation}
where $r_{ab}$ is the distance between atoms $a$ and $b$.


The implementation loops over all atoms in the system, independent whether they
are bonded or not. Therefore, this should primarily be used for e.g. metal
clusters. The output parameters are in the order $A(a,b)$, $p(a,b)$, $r_0(a,b)$,
$\chi(a,b)$ and $q(a,b)$ for each possible pair in atomic units. The
implementation does feature an analytical gradient for both coordinates and
parameters for enhanced performance of the local optimizations.

%Default borders for parameter optimizations are ....
%TODO 

\subsection{Adaptive LJ potential}

\subsection{Adaptive Morse potential}
%TODO document borders and caching
The adaptive Morse potential can be called using
\texttt{adaptivemorse:CLOSEBLOW,DISTBLOW}, where \texttt{CLOSEBLOW} is a blow 
factor defining when to use a cutoff for close cases and \texttt{DISTBLOW}
defines when to neglect contributions.

The implementation loops over all atoms in the system, independent whether they
are bonded or not. Therefore, this should primarily be used for e.g. metal
clusters. The output parameters are in the order $D_e$, $a$ and $r_e$ for each
possible pair in atomic units. The implementation does feature an analytical
gradient for both coordinates and parameters (suboptimal implementation) for
enhanced performance of the local optimizations.

Please note that currently no caching is implemented in the Morse potential. If
you want to use this potential for bigger systems, please contact the author(s)
whether this can be implemented.

\subsection{Benchmark: Ackley}

\subsection{Benchmark: Schaffer F6}

\subsection{Benchmark: Schaffer F7}

\subsection{Benchmark: Lunacek}

\subsection{goLPS parametrization using PROFESS}
\ogo{} can be used to fit local pseudopotentials following the goLPS formalism. Our
current implementation requires PROFESS (open-source) for the actual OFDFT calculations
with the modified pseudopotential and a set of helper scripts designed to translate
from \ogo{} output to PROFESS input and from PROFESS output back to \ogo{} input.
We will document here only the data exchange protocol, not the scripts themselves.
The current recommendation is to write them based on the problem to be fitted yourself.
If assistance with this step is required, please contact the author(s) of the
original goLPS publication.

In the following, the properties to-be-fitted are individually discussed. Please note
that this feature is experimental and we do NOT recommend fitting goLPSss for multiple
atoms simulaneously.

\subsubsection{Energy}
If a reference contains the energy (i.e., total energy), a directory following the scheme
\texttt{professenergycalc\_RUNNINGNUMNER\_TIMESTAMP} will be created. In it, the following files will be created:
\begin{itemize}
 \item a file \texttt{pseudopot\_ATOMNAME.recpot} for each ATOMNAME to be optimized,
 containing the current goLPS trial in PROFESS format
 \item a file \texttt{atompos.xyz} containing the atom positions of the reference in
 xyz format.
\end{itemize}
The following files will be copied from the directory in which \ogo{} is executed
\begin{itemize}
 \item \texttt{profess\_energy\_calc.sh} a script to be called (name can be overridden
 by setting the environment variable \texttt{OGO\_PROFESSENERGYSCRIPT} to the wanted
 name).
\end{itemize}

Subsequently, the script will be executed, using \texttt{XXX} as the first argument. When execution of
this script finishes, \ogo{} expects the calculated energy (in the same units as the reference) as
a simple number in a file \texttt{energy.dat}).

\subsubsection{Energy order}
If a reference contains the energy order of different atomic arrangements (e.g., crystal structures), for each crystal structure a directory following the scheme
\texttt{professenergycalc\_RUNNINGNUMNER\_TIMESTAMP} will be created. In it, the following files will be created:
\begin{itemize}
 \item a file \texttt{pseudopot\_ATOMNAME.recpot} for each ATOMNAME to be optimized,
 containing the current goLPS trial in PROFESS format
 \item a file \texttt{atompos.xyz} containing the atom positions of the reference in
 xyz format.
\end{itemize}
The following files will be copied from the directory in which \ogo{} is executed
\begin{itemize}
 \item \texttt{profess\_energy\_calc.sh} a script to be called (name can be overridden
 by setting the environment variable \texttt{OGO\_PROFESSENERGYSCRIPT} to the wanted
 name).
\end{itemize}

Subsequently, the script will be executed for each structure, using the defined key for this arrangement as the first argument. When execution of
this script finishes, \ogo{} expects the calculated energy (in the same units as the reference) as
a simple number in a file \texttt{energy.dat}). It will internally rank the energies.

\subsubsection{Bulk modulus}
If a reference contains the bulk modulus, a directory following the scheme
\texttt{professbulkcalc\_RUNNINGNUMNER\_TIMESTAMP} will be created. In it, the following files will be created:
\begin{itemize}
 \item a file \texttt{pseudopot\_ATOMNAME.recpot} for each ATOMNAME to be optimized,
 containing the current goLPS trial in PROFESS format.
\end{itemize}
The following files will be copied from the directory in which \ogo{} is executed
\begin{itemize}
 \item \texttt{profess\_bulk\_calc.sh} a script to be called (name can be overridden
 by setting the environment variable \texttt{OGO\_PROFESSBULKSCRIPT} to the wanted
 name).
\end{itemize}

Subsequently, the script will be executed, using the crystal symmetry (e.g., fcc) of the reference as the first argument. When execution of
this script finishes, \ogo{} expects the calculated bulk modulus (in the same units as the reference) as
a simple number in a file \texttt{bulk.modulus}).

\subsubsection{Equilibrium volume}
If a reference contains the equilibrium volume, a directory following the scheme
\texttt{professcellvolcalc\_RUNNINGNUMNER\_TIMESTAMP} will be created. In it, the following files will be created:
\begin{itemize}
 \item a file \texttt{pseudopot\_ATOMNAME.recpot} for each ATOMNAME to be optimized,
 containing the current goLPS trial in PROFESS format.
\end{itemize}
The following files will be copied from the directory in which \ogo{} is executed
\begin{itemize}
 \item \texttt{profess\_cellvol\_calc.sh} a script to be called (name can be overridden
 by setting the environment variable \texttt{OGO\_PROFESSCELLVOLSCRIPT} to the wanted
 name).
\end{itemize}

Subsequently, the script will be executed, using the crystal symmetry (e.g., fcc) of the reference as the first argument. When execution of
this script finishes, \ogo{} expects the calculated equilibrium volume (in the same units as the reference) as
a simple number in a file \texttt{cell.volume}).

\subsubsection{Combined energy, bulk modulus, and equilibrium volume calculations}
If the same reference contains total energy, bulk modulus, and equilibrium volume, they
will be evaluated simulaneously. A directory following the scheme
\texttt{professcombcalc\_RUNNINGNUMNER\_TIMESTAMP} will be created. In it, the following files will be created:
\begin{itemize}
 \item a file \texttt{pseudopot\_ATOMNAME.recpot} for each ATOMNAME to be optimized,
 containing the current goLPS trial in PROFESS format
 \item a file \texttt{atompos.xyz} containing the atom positions of the reference in
 xyz format.
\end{itemize}
The following files will be copied from the directory in which \ogo{} is executed
\begin{itemize}
 \item \texttt{profess\_combined\_calc.sh} a script to be called (name cannot be
 overridden).
\end{itemize}

Subsequently, the script will be executed, using the crystal symmetry (e.g., fcc) of the reference as the first argument. When execution of
this script finishes, \ogo{} expects the calculated equilibrium volume (in the same units as the reference) as
a simple number in a file \texttt{cell.volume}), the calculated bulk modulus (in the same units as the reference) as
a simple number in a file \texttt{bulk.modulus}), and the calculated energy (in the same units as the reference) as
a simple number in a file \texttt{energy.dat}).

\subsubsection{$\delta$ gauge}
If a reference contains the $\delta$ gauge, a directory following the scheme
\texttt{professdeltagaugecalc\_RUNNINGNUMNER\_TIMESTAMP} will be created. In it, the following files will be created:
\begin{itemize}
 \item a file \texttt{pseudopot\_ATOMNAME.recpot} for each ATOMNAME to be optimized,
 containing the current goLPS trial in PROFESS format.
\end{itemize}
The following files will be copied from the directory in which \ogo{} is executed
\begin{itemize}
 \item \texttt{profess\_deltagauge\_calc.sh} a script to be called (name can be overridden
 by setting the environment variable \texttt{OGO\_PROFESSDELTAGAUGESCRIPT} to the wanted
 name).
\end{itemize}

Subsequently, the script will be executed, using the crystal symmetry (e.g., fcc) of the reference as the first argument. When execution of
this script finishes, \ogo{} expects the calculated $\delta$ gauge (in the same units as the reference) as
a simple number in a file \texttt{Delta.out}).

\subsubsection{Forces}
If a reference contains nuclear forces, a directory following the scheme
\texttt{professforcescalc\_RUNNINGNUMNER\_TIMESTAMP} will be created. In it, the following files will be created:
\begin{itemize}
 \item a file \texttt{pseudopot\_ATOMNAME.recpot} for each ATOMNAME to be optimized,
 containing the current goLPS trial in PROFESS format
 \item a file \texttt{atompos.xyz} containing the atom positions of the reference in
 xyz format.
\end{itemize}
The following files will be copied from the directory in which \ogo{} is executed
\begin{itemize}
 \item \texttt{profess\_forces\_calc.sh} a script to be called (name can be overridden
 by setting the environment variable \texttt{OGO\_PROFESSFORCESSCRIPT} to the wanted
 name).
\end{itemize}

Subsequently, the script will be executed, using \texttt{XXX} as the first argument. When execution of
this script finishes, \ogo{} expects the calculated nuclear forces (in the same units as the reference) as
a simple number in a file \texttt{forces.out}). The format is as follows: first five lines are ignored, sixth line
onwards contains the forces (same order as reference) in columns two (x-component) to four (z-component).

\subsubsection{Electron density}
If a reference contains the electron density, a directory following the scheme
\texttt{professdensitycalc\_RUNNINGNUMNER\_TIMESTAMP} will be created. In it, the following files will be created:
\begin{itemize}
 \item a file \texttt{pseudopot\_ATOMNAME.recpot} for each ATOMNAME to be optimized,
 containing the current goLPS trial in PROFESS format
 \item a file \texttt{atompos.xyz} containing the atom positions of the reference in
 xyz format.
\end{itemize}
The following files will be copied from the directory in which \ogo{} is executed
\begin{itemize}
 \item \texttt{profess\_density\_calc.sh} a script to be called (name can be overridden
 by setting the environment variable \texttt{OGO\_PROFESSDENSITYSCRIPT} to the wanted
 name).
\end{itemize}

Subsequently, the script will be executed, using the defined tag for this reference as the first argument. When execution of
this script finishes, \ogo{} expects the calculated energy (in the same units as the reference) as
a simple number in a file \texttt{density.out}). The format is as follows (please note, this is PROFESS's fault!): a single (!), white space separated line, containing the grid dimensions as values two, four, six, and density values (same order as reference) as values 11 and later. Negative density values will be ignored.

\subsection{Interface to Polaris}
A word of prelimenary caution is needed here. This interface was written for a
single and special purpose. Therefore, it will not work for general-purpose
tasks by design.

The interface to Polaris is called through \texttt{AdaptivableChoice=polaris},
followed by the choices \texttt{:totalenergy} or \texttt{:waterinteraction}
defining which energy holds as reference. The references need to be adjusted
accordingly and must be in $E_h$.

Additionally, some auxiliary input and file stubs are required. Both a
\texttt{MET.FFM} and \texttt{REP.FFM} file need to be present in the input
directory, where all parameters that are to be fitted are replaced by the
string \texttt{OGOPARAM}. A reference \texttt{donee} file needs to be present,
starting from
\begin{verbatim}
-----------------------
|   Options energie   |
-----------------------
\end{verbatim}
and named \texttt{donee-ref.aux}. If there are other files to be copied for the
Polaris jobs, they need to be present in the input directory and their name
needs to be mentioned in a file \texttt{adaptive-polaris.aux} of form
\begin{verbatim}
###OGOLEMAUX###
filenametocopy1
filenametocopy2
\end{verbatim}

Since Polaris requires more than just a simple xyz-file as input, we assume that
for each reference point the following files exist:
\begin{itemize}
  \item \texttt{GEOM.xyz\_NUMBER} and
  \item \texttt{GEOM.sle\_NUMBER}
\end{itemize}
where number is the the position in the references starting from zero.

It is highly recommended to use a custom border file as explained above where
the parameters start with the \texttt{MET.FFM} parameters followed by the
\texttt{REP.FFM} parameters. Due to the parameters not being known by \ogo{} in
advance, the borders and parameter values in \ogo{} are in the same units as in
Polaris.

Last but not least, the path to the Polaris command needs to be exported
through an environment variable named \texttt{OGO\_POLARISCMD}.

Again, this interface was written for a single usage scenario and is not
designed to be used for other purposes.

\section{Global optimization algorithms}
All global optimization algorithms are chosen through the
\texttt{ParameterGlobOpt=} keyword. The input syntax follows the grammar as 
described in Section \ref{sec:geomglobopt}, it is 
\texttt{parameters{xover()mutation()}}.

Possible crossover choices are:
\begin{itemize}
  \item \texttt{germany:} a basic single cut, Gaussian distributed genotype 
crossover. Valid options:
  \begin{itemize}
    \item \texttt{gausswidth=X.X} Sets the width of the Gaussian to 
\texttt{X.X}. Default: 0.3.
  \end{itemize}
  \item \texttt{portugal:} a genotype, normal-distributed crossover. 
  \begin{itemize}
    \item \texttt{nocuts=X} sets the number of genotype cuts to \texttt{X}. 
Default: 1.
  \end{itemize}
  \item \texttt{noxover:} disables crossover operations.
  \item \texttt{multiple:} uses multiple crossover operators. Syntax identical 
to the one described in Section \ref{sec:geomglobopt}.
  \item \texttt{chained:} chains multiple crossover operators. Syntax identical 
to the one described in Section \ref{sec:geomglobopt}.
  \item \texttt{mutationasxover:} uses a mutation operator instead of a 
crossover operator for the xover step. Keyword must be followed by a valid 
mutation as described below.
\end{itemize}


After some prelimenary testing we are of the opinion that the exact crossover
(when using local optimization steps) is not important. Therefore we recommend
to use either \texttt{germany} or \texttt{portugal} with a low number of cuts.

In some cases (e.g. for a hypersurface with a lot of minima) it seems to be a
good idea to aggressively mutate the parameters, making \texttt{france} or
\texttt{spain} with a low number of cuts the methods of choice. 

Possible mutation choices are:
\begin{itemize}
  \item \texttt{germany:} a basic genotype mutation. Valid options:
  \begin{itemize}
    \item \texttt{mode=} sets the strength of the mutation. Valid input: 
\texttt{all} or \texttt{one}. Default: \texttt{one}.
  \end{itemize}
  \item \texttt{nomutation:} disables mutation operations.
  \item \texttt{multiple:} uses multiple mutation operators. Syntax identical 
to the one described in Section \ref{sec:geomglobopt}
  \item \texttt{chained:} chains multiple crossover operators. Syntax identical 
to the one described in Section \ref{sec:geomglobopt}.
\end{itemize}


\subsection{HACTAR}
Please see Section \ref{sec:geomhactar} for a description of the input format.

\section{Local optimization algorithms}
All local optimization algorithms are chosen through the
\texttt{ParameterLocOpt=} keyword. In general, the same choices are available 
as mentioned in Sec.~\ref{sec:geomlocopt}. Please note that the same 
restrictions concerning the methods obtained through the NUMAL interface apply 
as mentioned there.

In difference to the cluster structure optimization, the \texttt{backend=} 
definition present in the local optimization definitions has a somewhat 
restricted feature set. Either, one can specify \texttt{alldefault} to default 
to the classic setup of the fitness function. Note however, this option may 
well chance in the future! More future-proof is a pre-definition of fitness 
functions as defined in Sec.~\ref{sec:paramfitfuncdef}.

\section{Defining fitness functions}
\label{sec:paramfitfuncdef}
Fitness functions can be defined in between \texttt{<PARAMFITNESSFUNCTION>} 
tags. Per fitness function definition, one set of tags is required. 
Configuration options inside the tags are:
\begin{itemize}
  \item \texttt{FitFunctionTag=XYZ} assigns tag \texttt{XYZ} to this fitness 
function. Mandatory.
  \item \texttt{FitnessFunction=} specifies the fitness function to be used. At 
the moment, only \texttt{default} is allowed. Mandatory.
  \item \texttt{FitnessFunctionStyle=} specifies the fitness function style to 
be used. \texttt{full} choses all parameters to be exposed always. 
\texttt{ranged:} followed by range definitions (example: 1/3-4/7-8) specifies 
only these parts of the parameters to be exposed. Optional, default: 
\texttt{full}.
\end{itemize}

\section{Parameter output format}
Both the inital and final population are written in a human-readable format.
This format, as most input and output formats of \ogo{} uses an XML-like
formatting. The general format is as follows
\begin{verbatim}
<PARAMETERSFOR>
methodID
</PARAMETERSFOR>
<PARAMETERFITNESS>
1.23E-1
</PARAMETERFITNESS>
<PARAMETERS>
<ATOM>
Key
</ATOM>
<VALUES>
 -0.4636119053304735
 -0.5097016790211515
 -1.5220016461819896
 -0.5629664946696742
 -3.541843106605196
 -3.5394934242499287
 0.5378639455382299
 -3.5096479279350183
 0.2614123652818187
 -2.399411743051027
 -3.51017085778294
 1.4941457394022188
 -2.539798139301017
 -4.5084460220195215
 -0.5152036256889749
 -3.511654320068149
 -4.456450012737577
 -3.52578457936217
 -2.539843561431608
 1.477287999076277
</VALUES>
</PARAMETERS>
\end{verbatim}
where the method ID resembles the fitted potential. The parameter fitness is in
hartree and a \emph{reasonable} fitness depends on the settings, the
dimensionality and form of the problem. E.g. the fitness of a high dimensional
problem with a lot of reference points employing a steep penalty function
accepting only a small deviation from the reference is surely higher than, e.g.,
a two-body LJ(6,12,6) fit.

In general, as pointed out before, the best fitness is \emph{per definitionem}
0.0. This is, except for benchmark functions, in general not reachable.

This is followed by (depending on the potential and reference data) a number of
\texttt{<PARAMETER>} tags each uniquely identified by the key in between the
\texttt{<ATOM>} tags. The tag is somewhat misleading since depending on the
potential, it might not be a atom but, e.g., a three-body interaction. The
actual key transparently informs about such cases. Each \texttt{<PARAMETER>}
section contains in between the \texttt{<VALUES>} tags all raw values for the
specific key. Which parameter means what depends on the potential and is
explained in detail in the corresponding section of this manual.

Although we consider this format to be rather final, we do not guarantee that it
stays unchanged in later revisions of \ogo{}.

\section{Running the job}
After calling, e.g.,
\begin{verbatim}
java -jar ogolem.jar --adaptive input.ogo 2
\end{verbatim}
Depending on the adaptivable backend chosen, there might be a lot of files
appearing in the directory and the specified amount of CPUs be used. Concerning
the latter: it is not recommended to use all CPUs on a workstation which is
supposed to be still used for something else.

\ogo{} will clean up all automagically generated input files and/or directories
for other programs in case of a sucessfull call to that program. In case that
something unexpected (e.g. failing convergence) happens, the files will be left
for inspection.

Interesting datafiles are the intermediate binary pool
(\texttt{IntermediateParamPool.bin}) as well as the human-readable parameter
initial files \texttt{locrankRANKparamID.prm} and the final files
\texttt{rankRANKparamID.prm}, where \texttt{RANK} is the rank of the parameter
set (0 being the global minimum) and \texttt{ID} denotes when the specified set
was found.

\ogo{} might throw warnings or errors. These will be written to System.out and
System.err and are of interest.


\section{Parameter analysis}
\ogo{} provides some possibilities for parameter analysis. The most useful
application of them to-date is probably the intermediate check of parameters and
genetic progress at runtime. For this, you call the
\texttt{ogolem.jar} with the
\begin{verbatim}
--parameters
\end{verbatim}
option.

The analysis program needs a binary pool, either the final
\texttt{paramPool.bin}
or the temporary \texttt{IntermediateParamPool.bin}. This is specified using the
\texttt{-pool=} flag directly followed by the filename, e.g.
\begin{verbatim}
java -jar ogolem.jar --parameters -pool=IntermediateParamPool.bin
\end{verbatim}

Another option is possible
\begin{itemize}
  \item \texttt{-dontgetparams}\\
  Do not write the parameters out. Currently a rather useless option.
\end{itemize}

Once more, we want to encourage you here to contribute to the \ogo{} by
proposing some analysis techniques!

\chapter{Molecular design: Switchable molecules}
\textbf{IMPORTANT:\\
This part of the manual is not complete yet!}
%TODO

\section{Switch analysis}
\ogo{} provides means to check a running switch design job using the written
restart files. For this, you call the
\texttt{ogolem.jar} with the
\begin{verbatim}
--beswitched
\end{verbatim}
option.

The analysis program needs a binary pool, either the final
\texttt{switchespool.bin}
or the temporary \texttt{snapswitchpool.bin}. This is specified using the
\texttt{-i} flag followed by whitespace, e.g.
\begin{verbatim}
java -jar ogolem.jar --beswitched -i snapswitchpool.bin
\end{verbatim}

Some other options are possible
\begin{itemize}
  \item \texttt{-collblow}\\
	the blow factor for the collision detection (needed for sidechain
glueing), defaults to 1.4.
  \item \texttt{-getswitches}\\
	 create a directory named \texttt{switches} and write the switched of
this pool there in the human-readable format.
\end{itemize}


\chapter{Molecular design: General molecular design}
\textbf{IMPORTANT:\\
This part of the manual is not complete yet!}
%TODO

\chapter{Meta-benchmarking \textsc{ogolem} using Lionbench}
\section{Introduction}
Lionbench is a Scala-code developed by Marvin Kammler and Johannes Dieterich at 
the Georg-August-University G\"ottingen and has been embedded into the 
ogolem.jar. It is designed to automatically create \textsc{ogolem} input, run 
and analyze the performance based on all combinations of input settings. It is 
therefore possible to e.g. determine optimal input settings for a series of 
smaller cluster structure optimizations to use them for larger optimizations 
and save considerable amounts of computing time. Also, it allows to benchmark 
new algorithms against a set of known problems and in comparison to other 
algorithms.

Currently, no set of benchmark problems exists. Any contribution in this 
respect is very welcome!

\section{Usage within \textsc{ogolem}}
Access to Lionbench is simply realized by calling the ogolem.jar with the 
\texttt{--lionbench} flag followed by the Lionbench configuration file and the 
number of CPUs to be used for all \ogo runs.

Lionbench is calling ogolem through a runOGOLEM file which must be present and 
executable. It automatically sniffs through all subdirectories of the directory 
it has been started in and searches for \ogo{} input files to run them.

\section{Input file format: Main Lionbench configuration}
In general, the Lionbench configuration file contains keywords followed by 
whitespace and the option(s).
\begin{itemize}
 \item \texttt{NumberOfSamples}\\ The number of independent runs on each input 
configuration for statistics. Default: 5.
 \item \texttt{ReadLocalOpts}\\ If the number of local optimizations should be 
read. Default: false.
 \item \texttt{ExitOnFirstNotFound}\\ If Lionbench should exit when a global 
minimum is not found. Default: true.
 \item \texttt{ExitOnProblem}\\ If Lionbench should exit ungracefully upon the 
first encountered problem. Default: true.
 \item \texttt{Blacklist}\\ A comma-separated list of subdirectories that shall 
be ignored.
 \item \texttt{PartialBlacklist}\\ A comma-separated list of to-be-ignored 
inputs.
 \item \texttt{Whitelist}\\ A comma-separated list of subdirectories that shall 
be used (inverts the \texttt{Blacklist})
 \item \texttt{VerbosePerFolder}\\ If verbose output for every folder of tests 
is wanted. Default: true.
 \item \texttt{VerbosePerInput}\\ If verbose output for every test input is 
wanted. Default: true.
 \item \texttt{DeleteAllFiles}\\ If files should be deleted after successful 
excecution. Default: true.
 \item \texttt{SaveErrorErrout}\\ If error and output stream of every \ogo{} 
run should be save. Default: true.
 \item \texttt{AllowedDiffToGlobMin}\\ By how much the solution found by any 
input configuration is allowed to differ from the known global minimum. The
known global minimum value is set in the .ogo-file with the
\texttt{AcceptableFitness} option (cf.~pp.~\pageref{page1:acceptableFitness}
and \pageref{page2:acceptableFitness}).
Default: 1E-6.
\end{itemize}


\section{Input file format: Individual \ogo{} and \texttt{.lion} input files}
Lionbench works based on a stub for an \ogo{} input (e.g. test.ogo) and a 
corresponding Lionbench file (test.lion) defining how input parameters for 
ogolem should be manipulated. For convinience, it is possible to have one .lion 
file for all \ogo{} inputs if they should be altered in the same way. This file 
must be named \texttt{generic.lion} and will be used if present.

Every \ogo{} input variable to be varied with Lionbench will be replaced by the 
\texttt{LIONVAR} identifier in the \texttt{.ogo} file, e.g., 
\texttt{GlobOptAlgo=LIONVAR}. It is possible to vary multiple variables at 
once, Lionbench will automatically test all variations of them. The 
\texttt{.lion} file contains information on how those variables should be 
substituted and which files are needed for this \ogo{} run to complete (e.g. 
coordinate files, auxiliary files,...). For example:

\begin{verbatim}
<LIONOPTS>
germany AND portugal:1 AND holland
</LIONOPTS>
<LIONFILES>
dummy.xyz
</LIONFILES>
\end{verbatim}
where the first \texttt{LIONVAR} marked input option will be replaced with the 
identifiers \texttt{germany}, \texttt{portugal:1} and \texttt{holland}. If more 
than one input option are to be changed in the same Lionbench run, their order 
in both the \texttt{.ogo} and the \texttt{.lion} file must match.

In case of numerical variables, another syntax is available. A start to end 
with increment can be specified as \texttt{XXX...YYY...ZZZ} where \texttt{XXX} 
is the numerical start value, \texttt{YYY} is the end value and \texttt{ZZZ} is 
the increment.

Every directory with \texttt{.ogo}/\texttt{.lion} combinations must contain an 
executable shell script run ogolem named \texttt{runOGOLEM}. As an example:
\begin{verbatim}
#!/bin/sh
$JAVA_HOME/bin/java -jar $OGOLEMDIR/ogolem.jar --adaptive $1 $2 > errout 2>&1
\end{verbatim}
with the usual call of the \texttt{ogolem.jar}, the correct switch for the job 
to run and taking the input parameters \$1 and \$2 for the input file name and 
the number of threads, respectively.


\section{Output format}
\emph{Please note that the current output format is rather verbose and will be 
subject to change in later versions of Lionbench!}

Depending on the chosen verbosity, Lionbench will report the average and 
standard deviation of each input combination to console. Also, individual 
reports for each run are possible. There is a Python-script written by Marvin 
Kammler available to parse and beautify the Lionbench output. Eventually, the 
output will be altered to provide a better out-of-the-box solution.

\chapter{Lid/Threshold capabilites}
\ogo features an implementation of Sch\"on and Sibanis lid/threshold algorithm
that can be used to study funnels of the energy landscape of clusters and
molecules. Please read the original publications for a description of this
elegant algorithm.

Our lid implementation is called with the \texttt{--lid} flag followed by the
path to the .ogo file and the number of threads you intent to use for the lid
search. The .ogo file must contain a geometry definition and a choice for the
local optimization algorithm (please make sure to use something that also
doubles as a backend!) and contains all settings for the lid run in
\texttt{<LIDCONFIG>} tags.

Relevant input keywords are
\begin{itemize}
 \item \texttt{ThreshMetaRuns=}\\ The number of meta runs for the lid algorithm
(read: how often the total lid cycle will be run). Default: 2.
 \item \texttt{ThreshUseKearsleyOverlap}\\ Choose Kearsley's aligner for
structural comparison. Works only with single molecules. Mutually exclusive with
\texttt{TheshUseHundtOverlap}. Default: off.
 \item \texttt{ThreshUseHundtOverlap}\\ Choose Hundt's aligner for structural
comparison. Mutually exclusive with \texttt{ThreshUseKearsleyOverlap}. Default:
off.
 \item \texttt{ThreshKearslerRMSD=}\\ Maximal RMSD in bohr for structures to be
considered equal. Default: 0.1.
 \item \texttt{ThreshOverlapCOnfig=}\\ Modifies the settings of Hundt's aligner.
See description there.
 \item \texttt{ThreshEnergyPrescreen=}\\ The energy threshold in Hartree for
structural comparison or, if a secondary aligner is used, for prescreening.
Default: 1E-4.
 \item \texttt{NoVerboseSerialization}\\ Disable storage of intermediate results
after every meta run. Default: on.
 \item \texttt{ThreshMoveclass=}\\ Choosing a move class for the MC steps of the
lid run. Options:
 \begin{itemize}
  \item \texttt{mcmove:} MC moves in the Cartesian coordinate space. mcmove must
be followed by comma separated choices of move mode (0: move one atom, 1: move
random number of atoms, 2: move all atoms) and the maximum move in bohr.
  \item \texttt{mcextmove:} MC moves of fragements in the COM/Euler space.
mcextmove must be followed by comma separated choices of move mode (0: move one
fragment, 1: move random number of fragements, 2: move all fragements), the
maximum move in the COM space in bohr and the maximum move in the Euler space as
a fraction of the coordinate defintion (e.g. 0.2 as 20\% of the Euler
coordinate).
  \item \texttt{mcintmove:} MC moves in the internal coordinate space. THIS MUST
BE USED ONLY FOR SYSTEMS CONTAINING ONE MOLECULE WITH A Z-MATRIX INPUT PRESENT
AND PROPAGATED IN DIHEDRALS! mcintmove must be followed by comma separated
choices of the usual mode definition and the maximum move in the dihedral space,
e.g., 0.2 for 20\% of the $-\pi$ to $+\pi$ defined dihedral space. This input
options relies on the previously mentioned \texttt{DOF} definition to allow
specific dihedrals for propagation.
 \end{itemize}
 \item \texttt{ThresholdProvider=}\\ Defines which form the threshold increase
should have. Currently, you must use \texttt{simple:} followed by a comma
separated list of the starting energy, the end energy and the linear energy
increment in Hartree (e.g. \texttt{simple:0.02,0.20,0.01} to go from 0.02 to 0.2
Hartree in 0.01 Hartree increments).
 \item \texttt{ThresholdConfig=}\\Can be used to modify the internals of the
threshold algorithm. Tokens are semicolon separated and may contain the
following keywords
 \begin{itemize}
  \item \texttt{threshruns=} the number of propagations at a certain lid level.
Default: 30.
  \item \texttt{nomoves=} the number of MC moves in one threshold run. Default:
50000.
  \item \texttt{freqanalysis} carry out a frequency analysis to check each
obtained structure if it is a real minimum. NOT WORKING YET!
 \end{itemize}
\end{itemize}

Additionally, at least one starting structure is required. The defintion is
enclosed in \texttt{<LIDMINIMA>} tags, containing one structure (represented by
the path to an xyz file) per line, any number of lines is supported.

When running the lid algorithm, intermediate data will be printed out (unless
disabled) with the \texttt{intermediate} suffix and the number of the meta run
finished. Ultimately, the final results will be printed. A folder containing all
structures found by the algorithm, a dot file representing the graph and the
same graph as an \ogo internal binary for later reuse.

\chapter{JVM-based distributed-memory parallelization (RMI)}
\label{rmi}

The cluster, parameter and switch optimization parts of \ogo{} can also use a 
JVM-based parallelization via RMI (remote method invocation). While possessing
some unique features of its own, its targeted use is replacing the MPI-based
parallelization layer. See Ref.~\citenum{rmi} for a full discussion of the
background rationale for this RMI parallelization mode, of its different modi,
and of its performance, including in particular demonstrations of its
excellent parallel scaling. The most important
benefits include the ability to dynamically add or subtract clients at will
during runtime, also resulting in almost total immunity against hardware
failures, and parallelization across arbitrarily heterogenous ad-hoc networks,
only needing a standard JVM on each node and standard TCP/IP connections
between them.

In order for this parallelization scheme to work, a standard java distribution 
must be used, there need be two open TCP ports dedicated on the server and 
(if applicable) proxy nodes. Please note that we do not support running 
multiple proxies or server and proxies on the same machine.

Also, some installation-dependent flags to the JVM must be passed for security 
policies, CLASSPATH settings etc. These are included in the startup syntax
discussions below but are not yet explained further in this version of this
manual. 

There are three possible modes: 
\begin{description}
\item[simple clients:] 1 server communicating directly with $N$ clients,
\item[proxies:] 1 server communicating with $M$ proxies, which in turn
  communicate with $N_M$ simple clients each,
\item[threading clients:] 1 server communicating with $N$ tread-parallel
  clients. 
\end{description}
Essential similarities and differences between these modes (fully explained in
Ref.~\citenum{rmi}) include the following:
\begin{itemize}
\item The clients in the ``simple clients'' and ``proxies'' modes are simple
  in several ways:
  \begin{itemize}
  \item tasks they receive for execution always consist of just one single
    global optimization step, i.e., crossover and mutation of one or two
    ``parents'', and local optimization of the resulting one or two
    ``children'';
  \item these tasks are executed serially (no parallelization);
  \item from the user, these clients do not receive any problem-specific input
    (which is provided at runtime by the server).
  \end{itemize}
  In contrast, 
  \begin{itemize}
  \item both proxies and threading clients receive chunks of tasks, consisting
    of 1,2,\ldots,$m$ global optimization steps;
  \item threading clients execute their task chunks in thread-parallel manner;
  \item and both proxies and threading clients need a full \texttt{ogo}-file
    as input.
  \end{itemize}
\item From the viewpoint of the RMI server, there is no difference in
  communication to proxies or to threaded clients. From the viewpoint of the
  simple clients, there is no difference in communication to proxies or
  directly to the server. %bxh: I am speculating here. Is this correct?
\end{itemize}
Threading clients can be understood as combining the roles of one proxy and
all the simple clients connected to this proxy into one single job that is
then executed in shared-memory/thread-based parallelization. Alternatively,
one proxy and all simple clients connected to it can be understood as a
threading client executed in RMI parallelization.

\section{Starting RMI job-hierarchies}

As things are set up inside \ogo{}, all three of the above modes of RMI-mode
hierarchies (simple clients, proxies, threading clients) consist of several
(at least two) separate \ogo{} jobs that need 
to be started separately, and jobs ``higher'' in the hierarchy (servers,
proxies) always need to be started before jobs ``lower'' in the hierarchy
(proxies, clients). 

\subsection{Starting the RMI server}

Hence, in every mode, 
first a server process must be started. This can be accomplished using the
following (in one single, long line):
\begin{verbatim}
java -Djava.security.main -Djava.rmi.server.codebase=file:<path/to/ogolem-jar-file> 
-Djava.security.policy=polfile.txt -Djava.rmi.server.hostname=<IP-address> 
-jar <path/to/ogolem-jar-file> 
--rmiserver 
--jobtype <job type> 
--inputfile <input.ogo> 
--timeout <timeout> 
--timeoutjob <timeoutjob> 
--myserverport <server-port#> 
--myregistryport <registry-port#> 
--noproxies <#proxies>
--commname <nameofregistryobject>
\end{verbatim}
where
\begin{itemize}
  \item \texttt{-Djava.security.main} enables Java's security manager (mandatory)
  \item \texttt{-Djava.rmi.server.codebase=file:} tells the security manager where
    the ogolem.jar lies (mandatory)
  \item \texttt{-Djava.security.policy=} specifies the location of the security
    policy file (mandatory)
  \item \texttt{polfile.txt} is the filename of a java security policy file,
    with a content like this one (as the simplest and most permissive version):
\begin{verbatim}
grant{
        permission java.security.AllPermission;
};
\end{verbatim}
    We strongly advice you to only grant a minimum of permissions!
  \item \texttt{-Djava.rmi.server.hostname=} sets the internally used hostname
    (optional, use if there are problems with name resolution)
  \item \texttt{IP-address} is a standard IP address for the machine this
    server job is running on, e.g., a fully qualified numerical IPv4 address
    like \texttt{121.232.212.202}, or a string-based one like
    ``machine.domain.org'' or possibly even ``localhost'' (depending on the
    IP-setup of this machine);
  \item \texttt{job type} is the optimization problem to be solved. Options 
    are: \texttt{geom} (cluster structure optimization), \texttt{param}
    (parameter fitting), \texttt{switch} (switch design), this flag is
    mandatory;
  \item \texttt{input.ogo} is a regular \ogo{} input file for the problem to
    be optimized (mandatory);
  \item \texttt{timeout} is a timeout in seconds after which a client will be 
    purged from the server-internal list of active clients, provided there has
    been no communication with or heart beat from this client within this time
    interval. Default: 10s. Can be increased on stable networks. (optional);
  \item \texttt{timeoutjob} is a timeout in seconds after which a job will be 
    purged. This should correspond to the expected time a global optimization
    step (for simple clients) or a chunk of steps (proxies, threading clients)
    needs for normal execution, hence this timeout must be increased for
    demanding problems. Default: 10s. (optional);
  \item \texttt{server-port\#} is the IP port number that the server will use
    for task communication with the clients, typically/default 2500 (optional);
  \item \texttt{registry-port\#} is the IP port number where the RMI registry is
    running on, typically/default 1099 (optional);
  \item \texttt{\#proxies} is the number of proxies or threading clients that
    are initially expected to talk to this server. If there will be no
    proxies, specify -1 (optional).
    Note that \texttt{\#proxies} does \textit{not} limit (or otherwise affect)
    the number of actual proxies or threading clients that can connect to this
    server at runtime. \texttt{\#proxies = -1} denotes the ``simple client''
    mode, and \texttt{\#proxies >= 0} denotes a ``proxies'' or ``threading
    clients'' mode, where the initial value of \texttt{\#proxies} is only used
    to organize the initialization phase more sensibly.
  \item \texttt{commname\#} is the String name of the RMI communication for RMI 
    object export it must match on server, proxy, and client, typically/default
    RMICommunication (optional);
\end{itemize}
After \verb|--rmiserver|, the following flags can occur in any order.

The server process will now setup and configure itself and listen to incoming 
connections on \verb|server-port#| from clients.

\subsection{Starting RMI proxies or threading clients}

Subsequently, proxy processes or threading clients can be started if wished.

The proxy invocation is as follows (again with everything in one line):
\begin{verbatim}
java -Djava.rmi.server.codebase=file:<path/to/ogolem-jar-file> 
-Djava.security.policy=client.policy 
-jar <path/to/ogolem-jar-file> 
--rmiproxy
--jobtype <job type>
--inputfile <input.ogo>
--server <IP-address>
--serverregistryport <registry-port#>
--myserverport <myserver-port#>
--registryport <myregistry-port#>
--timeout <timeout> 
--timeoutjob <timeoutjob> 
--sleeptime <sleeptime>
--maxchunk <chunked steps>
--mergeinds <#indivs2merge>
--commname <nameofregistryobject>
\end{verbatim}
where
\begin{itemize}
\item \texttt{IP-address}: As above; however, on a remote machine, the user
  should make sure beforehand (with commands like \texttt{ping},
  \texttt{telnet}, etc.) that the address provided here is actually resolved
  in the desired way; this is a mandatory flag;
\item \texttt{registry-port\#} is the port number of the registry on the
  server (optional). The additional \texttt{server-port\#} provided at server
  startup need not be re-specified here since it is communicated to the client
  from the server.
\item \verb|myserver-port#| and \verb|myregistry-port#| are for the
  ``server-like role'' this proxy plays towards the RMI clients to be
  associated with it, and hence are fully analogous to the corresponding flags
  in the actual server invocation, including their default values and optional
  status. Only if proxy and server are located on the same physical machine,
  the ports must differ between them;
\item \texttt{sleeptime} has a dual role:
  It specifies the time delay between ``heartbeat'' signals sent to the
  server, indicating that the proxy is still ``alive''. And it also
  is the time this proxy waits after contacting the
  server and finding that the server is busy (e.g., waiting for another
  client);
\item \texttt{chunked steps} is the number of global optimization steps packed
  together into one chunk of tasks communicated from the server to the proxy,
  and of results communicated back from the proxy to the server; actual
  chunks may be smaller during initialization or in the final phase.
\item \verb|#indivs2merge| indicates how many individuals of the proxy's pool
  (at the time results of a chunk are communicated back to the server) should
  be merged with the server's pool. Default: -1 (full pool) (optional).
  Note that the proxy pool size need not be equal to the server pool size,
  since the proxy has its own input (where a pool size is specified).
\end{itemize}
After \verb|--rmiproxy|, the following flags can occur in any order.

Note that both server and proxy jobs do not do any actual calculations (global
optimization steps) themselves, but initially they do parse their \ogo{} input
files. Hence, before simple or threading clients are started,
\begin{itemize}
\item there will be no optimization progress;
\item but server and/or proxy jobs should report successful input parsing,
\item or may exit prematurely in case of input parsing errors (in which case
  client jobs being started subsequently may fail to connect).
\end{itemize}
Please note that both proxies and threading clients may output log files,
binary snapshots etc. Hence, they should be started in a directory other than 
the server process to avoid random overwrites.

The threading client invocation is as follows:
\begin{verbatim}
java -Djava.rmi.server.codebase=file:<path/to/ogolem-jar-file> 
-Djava.security.policy=client.policy 
-jar <path/to/ogolem-jar-file> 
--rmithreader 
--jobtype <job type> 
--inputfile <input.ogo> 
--server <IP-address>
--serverregistryport <registry-port#>
--threads <#threads> 
--sleeptime <sleeptime>
--maxchunk <chunked steps>
--mergeinds <#indiv2merge>
--commname <nameofregistryobject>
\end{verbatim}
Several of these flags have already been explained above. The remaining ones
are: 
\begin{itemize}
\item \texttt{registry-port\#} is the port number of the registry on the
  server. The additional \texttt{server-port\#} provided at server startup need
  not be re-specified here since it is communicated to the client from the
  server. In contrast to a proxy, this threading client does not play a dual
  RMI role, hence further ports for interactions with RMI clients need not be
  specified. 
\item \texttt{\#threads} specifies the number of shared-memory-parallel threads
  this client should use on this machine. Note that a value of 1 (one) is
  also possible, which establishes a mixture of features from simple clients
  (serial execution) with those of threaded clients (chunking, separate input
  file).
\end{itemize}

\subsection{Starting RMI simple clients}

Lastly, one can start simple client processes. As explained above, in ``simple
client'' mode, the simple clients connect directly to the server. In
``proxies'' mode, there typically are several groups of simple clients, each
of which connects to one proxy, and from the viewpoint of this simple client
group, this proxy takes the role of a server.

As explained above, simple clients need only very little information from the
user, hence their invocation is very simple, too:
\begin{verbatim}
java -Djava.rmi.server.codebase=file:<path/to/ogolem-jar-file> 
-Djava.security.policy=client.policy 
-jar <path/to/ogolem-jar-file> 
--rmiclient 
--server <IP-address>
--sleeptime <sleeptime>
--serverregistryport <server-port#>
--taskwaittime <taskwaittime>
\end{verbatim}
with
\begin{itemize}
 \item \texttt{taskwaittime} the time in seconds the client will wait for the
   server upon receiving a wait signal before contacting again. Default: 5s
   (optional).
\end{itemize}
and all other options have already been explained above, and again all
options after \verb|--rmiclient| can appear in any order. Naturally,
\verb|--server| is mandatory, but the other options are optional. If proxies
are used, \verb|--server| and \verb|--serverregistryport| denote the IP
address and registry port of the desired proxy (which takes the role of the
server here); the ``actual server'' is then effectively invisible to the
simple clients.

\newpage
\chapter{FAQs}
\section{\ogo{} doesn't run and/or behaves crazy}
Ensure that your environment is properly setup. In particular reduce the input
to a minimal example and check the following things:
\begin{itemize}
  \item A JRE (preferably openJDK/JRE or Sun/Oracle-JDK/JRE) of at least version
7 exists and works on your system.
  \item If an external program is called, check whether is is properly installed
and all conditions specified in the corresponding section are met.
  \item  Check whether your input is sane (in particular blow factors for
global geometry optimizations) and makes sense.
  \item set DebugLevel= to 1 or higher and check the additional output for
hints.
\end{itemize}

\section{I have problems running an external program}
Most of the times, this is due to the fact that the program cannot be found in
the \$PATH environment variable (for unixoid systems). Setup a wrapper script

\begin{verbatim}
#!/bin/sh
PROGRAM $@
\end{verbatim}

in your \$HOME/bin directory named like the \texttt{PROGRAM} you want to call,
mark it executable and see if it works (you might need to setup necessary
environment variables and/or load modules in this script as well). Debug
statements might help to make sure the wrapper is being executed:

\begin{verbatim}
#!/bin/sh
echo "Calling external program..." >> $HOME/debug_ogolem
PROGRAM $@
echo "Finished calling external program." >> $HOME/debug_ogolem
\end{verbatim}

\section{I want to report a bug, how does it work and what is required?}
In case your data is not confidential (or you can reduce the data set such that it
is not confidential), please file a bug report to \texttt{developers@ogolem.org}, 
providing your input(s) and a description of the problems encountered. We will 
treat the data confidentially.

Once the bug has been found and fixed, you can obtain a new snapshot of the 
ogolem.jar from the \ogo{} homepage.

\section{Does \ogo{} loose or change features over time?}
In general, we strive to remain backwards compatible with regards to both 
features 
and input. However, we will remove features that have been superseded by 
superior solutions, with the same rationale applying to input formats. Either of 
the two has so far happened only most seldomly. If a regression should be 
introduced, please notify the authors as we will want to fix it.

%TODO more FAQs


\addcontentsline{toc}{chapter}{References}
\bibliography{manual.bib}

\end{document}
